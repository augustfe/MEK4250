\documentclass[a4paper,12pt]{article}
\usepackage{amsmath, amsthm, amsfonts, amssymb}
\usepackage{mathtools}
\usepackage{microtype}
\usepackage{geometry}
\usepackage{booktabs}
\usepackage{graphicx}
\usepackage{tikz}
\usepackage{caption}
\usepackage{subcaption}
% \geometry{margin=1in}

\usepackage{enumitem}

\usepackage{hyperref}
\usepackage{xcolor}
\hypersetup{ % this is just my personal choice, feel free to change things
    colorlinks,
    linkcolor={red!50!black},
    citecolor={blue!50!black},
    urlcolor={blue!80!black},
}

\colorlet{myred}{red!20}
\colorlet{myblue}{blue!20}
\colorlet{mypurple}{purple!40}
\colorlet{myorange}{orange!20}
\colorlet{myteal}{teal!35}

% Your existing style for exercises, copied from your code:
\newtheoremstyle{exerciseStyle}%
{}{}%
{\normalfont}%
{}%
{\bfseries}%
{}%
{ }%
{\thmname{#1}\thmnumber{ #2}} % The heading "Exercise 4.6", etc.

\theoremstyle{exerciseStyle}
\newtheorem{exerciseTemp}{Exercise}
% This "exerciseTemp" environment has a normal counter
% but we will override that counter each time we begin an exercise.

% Now define the "exercise" environment that takes a *manual* label
\NewDocumentEnvironment{exercise}{m}{%
    % 1. Reset the counter so we start from 0
    % 2. Define \theexerciseTemp to be whatever was passed in, e.g. "4.6"
    \setcounter{exerciseTemp}{0}%
    \renewcommand{\theexerciseTemp}{#1}%
    \begin{exerciseTemp}%
    }{%
    \end{exerciseTemp}
}

% Your existing style for exercises, copied from your code:
\newtheoremstyle{solutionStyle}%
{}{}%
{\normalfont}%
{}%
{\bfseries}%
{}%
{ }%
{\thmname{#1}\thmnumber{ #2}} % The heading "Solution 4.6", etc.

\theoremstyle{solutionStyle}
\newtheorem{solutionTemp}{Solution}
% This "solutionTemp" environment has a normal counter
% but we will override that counter each time we begin an solution.

% Now define the "solution" environment that takes a *manual* label
\NewDocumentEnvironment{solution}{m}{%
    % 1. Reset the counter so we start from 0
    % 2. Define \thesolutionTemp to be whatever was passed in, e.g. "4.6"
    \setcounter{solutionTemp}{0}%
    \renewcommand{\thesolutionTemp}{#1}%
    \begin{solutionTemp}%
    }{%
    \end{solutionTemp}
}

\DeclarePairedDelimiter\abs{\lvert}{\rvert}%
\DeclarePairedDelimiter\norm{\lVert}{\rVert}%

\title{
    MEK4250\\
    \small{First mandatory exercise}
}
\author{August Femtehjell}
\date{Spring 2025}

\begin{document}

\maketitle

\tableofcontents

\begin{abstract}
    This document contains my solutions to the exercises for the course MEK4250--Finite Elements in Computational Mechanics, taught at the University of Oslo in the spring of 2025.
    The code for everything, as well as this document, can be found at my GitHub repository: \url{https://github.com/augustfe/MEK4250}.
\end{abstract}

\section{Discretization of a convection-diffusion problem}

\begin{exercise}
    Show that the matrix obtained from a central difference scheme applied to the operator $L u = u_x$ is skew-symmetric.
    Furthermore, show that the matrix obtained by linear continuous Lagrange elements are also skew-symmetric.
    Remark: The matrix is only skew-symmetric in the interior of the domain, not at the boundary.
\end{exercise}

\begin{solution}
    We consider the operator $L u = u_x$.
    The central difference scheme applied to this operator is
    \begin{equation*}
        L_h u = \frac{u_{i+1} - u_{i-1}}{2h},
    \end{equation*}
    where $h$ is the mesh size.
    The matrix representation of this operator is, considering the interior points only,
    \begin{equation*}
        L_h = \frac{1}{2h}
        \begin{bmatrix}
            -1 & 0 & 1 \\
            & -1 & 0 & 1 \\
            && \ddots & \ddots & \ddots \\
            &&& -1 & 0 & 1
        \end{bmatrix}.
    \end{equation*}
    This matrix is skew-symmetric, as can be seen by transposing it and negating it, ignoring the mismatched dimension of the matrix caused by ignoring the boundaries.

    The matrix $A$ for describing $L u = u_x$ using linear continuous Lagrange elements is defined by the elements
    \begin{equation*}
        A_{ij} = \int_{\Omega} \phi_i' \phi_j \, dx.
    \end{equation*}
    As the basis functions are linear, the derivative of the basis functions are constant.
    If there is no overlap between the basis functions, the integral is zero, and we clearly have $A_{ij} = 0 = -A_{ji}$.
    Suppose then that they do contain some overlap, on an interval $[x_l, x_u]$.
    Then we have
    \begin{equation*}
        A_{ij} = \int_{\Omega} \phi_i' \phi_j \, dx = \int_{x_l}^{x_u} \phi_i' \phi_j \, dx = -\int_{x_l}^{x_u} \phi_i \phi_j' \, dx + \left[ \phi_i \phi_j \right]_{x_l}^{x_u} = -A_{ji} + \left[ \phi_i \phi_j \right]_{x_l}^{x_u}.
    \end{equation*}
    Now, as the exercise hints to, for the interior points, the boundary term is zero, and we have $A_{ij} = -A_{ji}$.
\end{solution}

\begin{exercise}
    Estimate numerically the constant in Cea’s lemma for various $\alpha$ and $h$ for the Example 4.1.
\end{exercise}

\end{document}