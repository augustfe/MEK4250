\begin{exercise}{5.6}
    Consider the eigenvalues of the operators, $L_1$, $L_2$, and $L_3$, where $L_1 u = u_x$, $L_2 u = -\alpha u_{xx}$, $\alpha = 1.0 e^{-5}$, and $L_3 = L_1 + L_2$, with homogeneous Dirichlet conditions.
    For which of the operators are the eigenvalues positive and real?
    Repeat the exercise with $L_1 = x u_x$.
\end{exercise}

\begin{solution}{5.6}
    Here, I assume that the domain is $[0, 1]$, and the boundary conditions are $u(0) = u(1) = 0$.
    Solving for the eigenvalues for the first operators is not terribly difficult, although it is a bit tedious.
    Instead, we will use the finite element method.

    With the eigenvalue problem $L u = \lambda u$, we have the associated weak form
    \begin{equation*}
        \int_0^1 L u \, v \, \diff x = \lambda \int_0^1 u \, v \, \diff x,
    \end{equation*}
    where $L$ is one of the operators $L_1$, $L_2$, or $L_3$.
    For the first operator, we then have
    \begin{equation*}
        \int_0^1 u_x \, v \, \diff x = \lambda \int_0^1 u \, v \, \diff x.
    \end{equation*}
    For $L_2$ we have
    \begin{equation*}
        \int_{0}^{1} -\alpha u_{xx} \, v \, \diff x = \lambda \int_{0}^{1} u \, v \, \diff x,
    \end{equation*}
    however, we can integrate by parts to get
    \begin{equation*}
        \int_{0}^{1} \alpha u_x \, v_x \, \diff x = \lambda \int_{0}^{1} u \, v \, \diff x.
    \end{equation*}
    Finally, for $L_3$ we have
    \begin{equation*}
        \int_{0}^{1} (u_x - \alpha u_{xx}) \, v \, \diff x = \lambda \int_{0}^{1} u \, v \, \diff x,
    \end{equation*}
    which again can be integrated by parts to get
    \begin{equation*}
        \int_{0}^{1} u_x \, v  + \alpha u_x \, v_x \, \diff x = \lambda \int_{0}^{1} u \, v \, \diff x.
    \end{equation*}

    This is implemented in \texttt{eigs.py}, resulting in the eigenvalues shown in \cref{fig:eigenvalues}.
    As we see, adding the diffusion term to the first operator results in $L_3$ having eigenvalues with positive real parts.

    \begin{figure}[ht]
        \centering
        \includegraphics[width=0.8\textwidth]{eigs.pdf}
        \caption{Eigenvalues for the operators $L_1$, $L_2$, and $L_3$ with homogeneous Dirichlet conditions. Computed with $N = 500$ first-order Lagrange elements.\label{fig:eigenvalues}}
    \end{figure}
\end{solution}

\begin{exercise}{6.1}\label{ex:6.1}
    Show that the conditions in \cref{eq:6.15,eq:6.16,eq:6.17} are satisfied for $V_h = H^1_0$ and $Q_h = L^2$.
\end{exercise}

\begin{solution}{6.1}
    For this exercise we are considering Stokes problem, wherein we want to find $u_h \in V_h$ and $p_h \in Q_h$ such that
    \begin{equation}
        \begin{split}
            a(u_h, v_h) + b(p_h, v_h) &= f(v_h), \quad \forall v_h \in V_h, \\
            b(q_h, u_h) &= 0, \quad \forall q_h \in Q_h,
        \end{split}
    \end{equation}
    where
    \begin{equation}
        \begin{split}
            a(u, v) &= \int_{\Omega} \nabla u : \nabla v \diff x, \\
            b(p, v) &= \int_{\Omega} p \, \nabla \cdot v \diff x, \\
            f(v) &= \int_{\Omega} f \, v \diff x + \int_{\Omega_N} h \, v \diff s.
        \end{split}
    \end{equation}
    The conditions to show are then
    \begin{align}
        \text{\small Boundedness of $a$:}
        && \quad a(u_h, v_h) &\leq C_1 \norm{u_h}_{V_h} \norm{v_h}_{V_h},
        & \forall u_h, v_h \in V_h,
        \tag{6.15} \label{eq:6.15}
        \\
        \text{\small Boundedness of $b$:}
        && \quad b(q_h, u_h) &\leq C_2 \norm{q_h}_{Q_h} \norm{u_h}_{V_h},
        & \forall u_h \in V_h, q_h \in Q_h,
        \tag{6.16} \label{eq:6.16}
        \\
        \text{\small Coercivity of $a$:}
        && \quad a(u_h, u_h) &\geq C_3 \norm{u_h}_{V_h}^2,
        & \forall u_h \in Z_h,
        \tag{6.17} \label{eq:6.17}
    \end{align}
    where $Z_h = \{ u_h \in V_h \mid b(u_h, q_h) = 0, \, \forall q_h \in Q_h \}$.

    For \cref{eq:6.15}, we have
    \begin{equation}
        a(u_h, v_h) = \int_{\Omega} \nabla u_h : \nabla v_h \diff x,
    \end{equation}
    which by the Cauchy-Schwarz inequality is bounded by
    \begin{equation}
        a(u_h, v_h) \leq \norm{\nabla u_h}_{L^2} \norm{\nabla v_h}_{L^2} = \abs{u_h}_{H^1} \abs{v_h}_{H^1}.
    \end{equation}
    As we have previously shown that the $H^1$ semi-norm is equivalent to the $H^1$ norm, \cref{eq:6.15} is satisfied.

    For \cref{eq:6.16}, we have
    \begin{equation}
        b(q_h, u_h) = \int_{\Omega} q_h \, \nabla \cdot u_h \diff x,
    \end{equation}
    which by the Cauchy-Schwarz inequality is bounded by
    \begin{equation}
        b(q_h, u_h) \leq \norm{q_h}_{L^2} \norm{\nabla \cdot u_h}_{L^2}.
    \end{equation}
    We already have $\norm{q_h}_{L^2}$ as desired, and we now need to relate $\norm{\nabla \cdot u_h}_{L^2}$ to $\norm{\nabla u_h}_{L^2}$.
    We have
    \begin{align*}
        \abs{\nabla \cdot u(x)}^2
        &= \abs*{\sum_{i = 0}^{d} \frac{\partial u_i}{\partial x_i}(x)}^2
        \leq \left(
            \sum_{i = 0}^{d} 1
        \right) \left(
            \sum_{i = 0}^{d} \left( \frac{\partial u_i}{\partial x_i}(x) \right)^2
        \right)
        = d \sum_{i = 0}^{d} \left( \frac{\partial u_i}{\partial x_i}(x) \right)^2.
    \end{align*}
    Therefore,
    \begin{align*}
        \norm{\nabla \cdot u_h}_{L^2}^2
        &= \int_{\Omega} \abs{\nabla \cdot u_h}^2 \diff x
        \leq d \int_{\Omega} \sum_{i = 0}^{d} \left(
            \frac{\partial u_i}{\partial x_i}
        \right)^2 \diff x \\
        &\leq d \int_{\Omega} \sum_{i = 0}^{d} \sum_{j = 0}^{d} \left(
            \frac{\partial u_i}{\partial x_j}
        \right)^2 \diff x
        = d \int_{\Omega} \abs{\nabla u_h}^2 \diff x \\
        &= d \norm{\nabla u_h}_{L^2}^2.
    \end{align*}
    Using this, we have
    \begin{equation}
        b(q_h, u_h) \leq \sqrt{d} \norm{q_h}_{L^2} \norm{\nabla u_h}_{L^2},
    \end{equation}
    and as such \cref{eq:6.16} is satisfied.

    For \cref{eq:6.17}, we have
    \begin{equation}
        a(u_h, u_h)
        = \int_{\Omega} \nabla u_h : \nabla u_h \diff x
        = \norm{\nabla u_h}_{L^2}^2 = \abs{u_h}_{H^1}^2.
    \end{equation}
    As we have previously shown that the $H^1$ semi-norm is equivalent to the $H^1$ norm, \cref{eq:6.17} is satisfied.

    We have thus shown that the conditions in \cref{eq:6.15,eq:6.16,eq:6.17} are satisfied for $V_h = H^1_0$ and $Q_h = L^2$.
\end{solution}

\begin{exercise}{6.2}
    Show that the conditions in \cref{eq:6.15,eq:6.16,eq:6.17} are satisfied for Taylor--Hood and Mini discretizations. % chktex 8
    (Note that Crouzeix--Raviart is non-conforming so it is more difficult to prove these conditions for this case.) % chktex 8
\end{exercise}

\begin{solution}{6.2}
    The Taylor--Hood elements are defined by the basis functions % chktex 8
    \begin{equation}
        \begin{split}
            u &: N_i = a_i + b_i x + c_i y + d_i x y + e_i x^2 + f_i y^2, \\
            p &: L_i = k_i + l_i x + m_i y,
        \end{split}
    \end{equation}
    under the condition that $N_i$ is continuous across element boundaries.
    As such, we have that $u_h \in V_h \subset H^1_0$ and $p_h \in Q_h \subset L^2$ for the Taylor--Hood elements. % chktex 8
    They therefore satisfy the conditions in \cref{eq:6.15,eq:6.16,eq:6.17}, as shown in \cref{ex:6.1}.

    The Mini elements are, as far as I understand, defined by the basis functions
    \begin{equation}
        \begin{split}
            u &: N_i = a_i + b_i x + c_i y + d_i xy(1 - x - y), \\
            p &: L_i = k_i + l_i x + m_i y,
        \end{split}
    \end{equation}
    and we now need to show that $u_h \in V_h \subset H^1_0$.
    % With a slight abuse of notation, we have
    % \begin{align*}
    %     \nabla \cdot N_i &=
    %     \nabla \cdot (a_i + b_i x + c_i y + d_i xy - d_i x^2 y - d_i x y^2) \\
    %     &= b_i + c_i - d_i x^2 - 4d_i xy - d_i y^2 \\
    %     &= b_i + c_i - d_i (x^2 + 4xy + y^2).
    % \end{align*}
    % Presumably, we have that $V_h \subset H^1_0$, however I don't understand Fortin's trick.
    Notice that along each
\end{solution}

\begin{exercise}{6.6}
    In the previous problem the solution was a second order polynomial in the velocity and first order in the pressure.
    We may therefore obtain the exact solution and it is therefore difficult to check order of convergence for higher order methods with this solution.
    In this exercise you should therefore implement the problem
    \begin{align*}
        u &= (\sin(\pi y), \cos(\pi x)),
        & p &= \sin(2 \pi x),
        & f &= -\Delta u - \nabla p.
    \end{align*}
    Test whether the approximation is of the expected order for $P_4$--$P_3$, $P_4$--$P_2$, $P_3$--$P_2$, and $P_3$--$P_1$.
\end{exercise}

\begin{solution}{6.6}
    The solver is implemented in \texttt{stokes.py}, and the results are shown in \cref{fig:convergence}.
    As we see, we get convergence of order 4, 3, 3, and 2 for the $P_4$--$P_3$, $P_4$--$P_2$, $P_3$--$P_2$, and $P_3$--$P_1$ elements, respectively.
    As to whether this is the expected order of convergence, I am not sure.

    \begin{figure}[htbp]
        \centering
        \includegraphics[
            width=0.8\textwidth,
            keepaspectratio,
        ]{stokes_convergence.pdf}
        \caption{Convergence of the $P_4$--$P_3$, $P_4$--$P_2$, $P_3$--$P_2$, and $P_3$--$P_1$ elements for the Stokes problem with the exact solution $u = (\sin(\pi y), \cos(\pi x))$ and $p = \sin(2 \pi x)$. Computed with $N = 4, 8, 16, 32, 64$.\label{fig:convergence}}
    \end{figure}
\end{solution}

\newpage

\begin{comment}

\begin{exercise}{5.6}
    Consider the eigenvalues of the operators, $L_1$, $L_2$, and $L_3$, where $L_1 u = u_x$, $L_2 u = -\alpha u_{xx}$, $\alpha = 1.0 e^{-5}$, and $L_3 = L_1 + L_2$, with homogeneous Dirichlet conditions.
    For which of the operators are the eigenvalues positive and real?
    Repeat the exercise with $L_1 = x u_x$.
\end{exercise}

\begin{comment}
    \begin{solution}{4.6}
    For this exercise, I'm assuming that the domain is $[0, 1]$.
    For $L_1$, the eigenfunctions are of the from
    \begin{equation}
        u(x) = C e^{\lambda x},
    \end{equation}
    as we then have
    \begin{equation}
        L_1 C e^{\lambda x} = \frac{\partial}{\partial x} C e^{\lambda x} = C \lambda e^{\lambda x}.
    \end{equation}
    For the boundary conditions, we have $u(0) = u(1) = 0$, which gives us
    \begin{equation}
        C e^{\lambda \cdot 0} = C e^0 = C
    \end{equation}
    and
    \begin{equation}
        e^{\lambda \cdot 1} = C e^{\lambda},
    \end{equation}
    which gives us the trivial solution $C = 0$ and $u(x) = 0$, for all $\lambda$.

    For $L_2$, we consider first the operator $\frac{\partial^2}{\partial x^2}$.
    The eigenfunctions of this operator are loosely of the form
    \begin{equation}
        u(x) = e^{i x},
    \end{equation}
    as then
    \begin{equation}
        \frac{\partial^2}{\partial x^2} e^{i x} = -i^2 e^{i x} = -e^{i x}.
    \end{equation}
    In order to get the factor $\alpha$, we need to adjust the exponent such that we have
    \begin{equation}
        u(x) = e^{i \sqrt{\alpha} x}.
    \end{equation}
\end{solution}

\begin{solution}{5.6}
    For this exercise, let the domain be $[0, 1]$ for simplicity.
    For the first part of the exercise, we are seeking solutions to
    \begin{equation}
        u_x = \lambda u,
    \end{equation}
    for which the solutions are
    \begin{equation}
        u(x) = C e^{\lambda x}.
    \end{equation}
    The boundary conditions are $u(0) = u(1) = 0$, which gives us that $C = 0$, and thus $u(x) = 0$ for all $\lambda$.

    For the second part of the exercise, we are seeking solutions to
    \begin{equation}
        -\alpha u_{xx} = \lambda u,
    \end{equation}
    or equivalently
    \begin{equation}
        -u_{xx} = \tfrac{\lambda}{\alpha} u.
    \end{equation}
    As the second derivative is a scaled version of the function itself, with opposite sign, we can guess that the solutions are of the form
    \begin{equation}
        u(x) = \sin(C x)
        \quad \text{or} \quad
        u(x) = \cos(C x).
    \end{equation}
    As the boundary conditions are homogeneous Dirichlet conditions, we can see that the sine function is the only one that satisfies the boundary conditions, for $C = k \pi$, $k = 0, 1, 2, \ldots$.
    Taking the double derivative of the sine function, we get
    \begin{equation}
        -u_{xx} = (k \pi)^2 \sin(k \pi x),
    \end{equation}
    and hence
    \begin{equation}
        \tfrac{\lambda}{\alpha} = (k \pi)^2 \implies \lambda = \alpha (k \pi)^2.
    \end{equation}
    As $\alpha > 0$, we see that the eigenvalues are positive and real for $L_2$, for $k \geq 1$.

    For $L_3$, we are looking for a solution
    \begin{equation}
        u_x - \alpha u_{xx} = \lambda u.
    \end{equation}
    Recalling back to videregående ODE solving, we can guess that the solution is of the form
    \begin{equation}
        u(x) = e^{rx},
    \end{equation}
    which gives the equation
    \begin{equation}
        -\alpha r^2 + r - \lambda = 0.
    \end{equation}
    Recalling the abc-formula, we then have the solutions
    \begin{align}
        r &=
        \frac{-1 \pm \sqrt{1 - 4 \alpha \lambda}}{-2 \alpha}
        = \frac{1 \mp \sqrt{1 - 4 \alpha \lambda}}{2 \alpha}.
    \end{align}
    For $1 - 4 \alpha \lambda > 0$, we have two real roots, however they only permit the trivial solution.
    For $1 - 4 \alpha \lambda = 0$, we have one real root, $r = 1 / 2\alpha$, which gives us a solution of the form
    \begin{equation}
        (C_1 + C_2 x) e^{x / 2\alpha},
    \end{equation}
    however for the boundary conditions, we must have $C_1 = C_2 = 0$, and hence the only solution is the trivial one.

    Finally, for $1 - 4 \alpha \lambda < 0$, we have two complex roots
    \begin{equation}
        r = \frac{1}{2\alpha} \pm i \frac{\sqrt{4 \alpha \lambda - 1}}{2\alpha},
    \end{equation}
    which gives us a solution of the form
    \begin{equation}
        e^{x / 2\alpha} \left(
            C_1 \cos\left(\frac{\sqrt{4 \alpha \lambda - 1}}{2\alpha} x\right)
            + C_2 \sin\left(\frac{\sqrt{4 \alpha \lambda - 1}}{2\alpha} x\right)
        \right).
    \end{equation}
    At $x = 0$, we have
    \begin{equation}
        e^{0} \left(
            C_1 \cos\left(\frac{\sqrt{4 \alpha \lambda - 1}}{2\alpha} \cdot 0 \right)
            + C_2 \sin\left(\frac{\sqrt{4 \alpha \lambda - 1}}{2\alpha} \cdot 0 \right)
        \right),
    \end{equation}
    such that we must have $C_1 = 0$.
    At $x = 1$, we have
    \begin{equation}
        e^{1 / 2\alpha} \left( C_2 \sin\left(\frac{\sqrt{4 \alpha \lambda - 1}}{2\alpha}\right) \right) = 0,
    \end{equation}
    which is valid when $\frac{\sqrt{4 \alpha \lambda - 1}}{2\alpha} = k \pi$, $k = 0, 1, 2, \ldots$.
    This gives us the eigenvalues
    \begin{align*}
        \frac{\sqrt{4 \alpha \lambda - 1}}{2\alpha} &= k \pi \\
        4 \alpha \lambda - 1 &= (2\alpha k \pi)^2 \\
        \lambda &= \frac{(2 \alpha k \pi)^2 + 1}{4 \alpha} \\
        \lambda &= \alpha (k \pi)^2 + \frac{1}{4 \alpha},
    \end{align*}
    which are always positive and real.

    Changing $L_1$ to $x u_x$, we have the equation
    \begin{equation}
        x u_x = \lambda u \implies u_x = \frac{\lambda}{x} u.
    \end{equation}
    As far as I can see, the monomials satisfy the equation, as
    \begin{equation}
        u_x = \frac{\partial}{\partial x} x^n = n x^{n - 1} = \frac{n}{x} x^n = \frac{n}{x} u.
    \end{equation}
    Looking even closer, we see that we do not even need to restrict ourselves the monomials, but can rather have
    \begin{equation}
        u(x) = C x^\lambda,
        \quad \text{for }
        \lambda \in \mathbb{R},
    \end{equation}
    however as zero is included in the domain, we require that $\lambda > 0$.
    At $x = 0$ we have $u(0) = 0$, while we at $x = 1$ have
    \begin{equation}
        u(1) = C,
    \end{equation}
    and as such we sadly only have the trivial solution $u(x) = 0$ for all $\lambda$.

    I have no clue as to how to find the eigenvalues for $L_3$ when $L_1 = x u_x$.

    With this new $L_1$, we have that
    \begin{equation}
        L_3 u = x u_x - \alpha u_{xx},
    \end{equation}

    \begin{equation}
        \lambda u - x u_x + \alpha u_{xx} = 0.
    \end{equation}
\end{solution}

\begin{exercise}{6.1}
    Show that the conditions in \cref{eq:6.15,eq:6.16,eq:6.17} are satisfied for $V_h = H^1_0$ and $Q_h = L^2$.
\end{exercise}

\begin{solution}{6.1}
    For this exercise we are considering Stokes problem, wherein we want to find $u_h \in V_h$ and $p_h \in Q_h$ such that
    \begin{equation}
        \begin{split}
            a(u_h, v_h) + b(p_h, v_h) &= f(v_h), \quad \forall v_h \in V_h, \\
            b(q_h, u_h) &= 0, \quad \forall q_h \in Q_h,
        \end{split}
    \end{equation}
    where
    \begin{equation}
        \begin{split}
            a(u, v) &= \int_{\Omega} \nabla u : \nabla v \diff x, \\
            b(p, v) &= \int_{\Omega} p \, \nabla \cdot v \diff x, \\
            f(v) &= \int_{\Omega} f \, v \diff x + \int_{\Omega_N} h \, v \diff s.
        \end{split}
    \end{equation}
    The conditions to show are then
    \begin{align}
        \text{\small Boundedness of $a$:}
        && \quad a(u_h, v_h) &\leq C_1 \norm{u_h}_{V_h} \norm{v_h}_{V_h},
        & \forall u_h, v_h \in V_h,
        \tag{6.15} \label{eq:6.15}
        \\
        \text{\small Boundedness of $b$:}
        && \quad b(q_h, u_h) &\leq C_2 \norm{q_h}_{Q_h} \norm{u_h}_{V_h},
        & \forall u_h \in V_h, q_h \in Q_h,
        \tag{6.16} \label{eq:6.16}
        \\
        \text{\small Coercivity of $a$:}
        && \quad a(u_h, u_h) &\geq C_3 \norm{u_h}_{V_h}^2,
        & \forall u_h \in Z_h,
        \tag{6.17} \label{eq:6.17}
    \end{align}
    where $Z_h = \{ u_h \in V_h \mid b(u_h, q_h) = 0, \, \forall q_h \in Q_h \}$.

    For \cref{eq:6.15}, we have
    \begin{equation}
        a(u_h, v_h) = \int_{\Omega} \nabla u_h : \nabla v_h \diff x,
    \end{equation}
    which by the Cauchy-Schwarz inequality is bounded by
    \begin{equation}
        a(u_h, v_h) \leq \norm{\nabla u_h}_{L^2} \norm{\nabla v_h}_{L^2} = \abs{u_h}_{H^1} \abs{v_h}_{H^1}.
    \end{equation}
    As we have previously shown that the $H^1$ semi-norm is equivalent to the $H^1$ norm, \cref{eq:6.15} is satisfied.

    For \cref{eq:6.16}, we have
    \begin{equation}
        b(q_h, u_h) = \int_{\Omega} q_h \, \nabla \cdot u_h \diff x,
    \end{equation}
    which by the Cauchy-Schwarz inequality is bounded by
    \begin{equation}
        b(q_h, u_h) \leq \norm{q_h}_{L^2} \norm{\nabla \cdot u_h}_{L^2}.
    \end{equation}
    We already have $\norm{q_h}_{L^2}$ as desired, and we now need to relate $\norm{\nabla \cdot u_h}_{L^2}$ to $\norm{\nabla u_h}_{L^2}$.
    We have
    \begin{align*}
        \abs{\nabla \cdot u(x)}^2
        &= \abs*{\sum_{i = 0}^{d} \frac{\partial u_i}{\partial x_i}(x)}^2
        \leq \left(
            \sum_{i = 0}^{d} 1
        \right) \left(
            \sum_{i = 0}^{d} \left( \frac{\partial u_i}{\partial x_i}(x) \right)^2
        \right)
        = d \sum_{i = 0}^{d} \left( \frac{\partial u_i}{\partial x_i}(x) \right)^2.
    \end{align*}
    Therefore,
    \begin{align*}
        \norm{\nabla \cdot u_h}_{L^2}^2
        &= \int_{\Omega} \abs{\nabla \cdot u_h}^2 \diff x
        \leq d \int_{\Omega} \sum_{i = 0}^{d} \left(
            \frac{\partial u_i}{\partial x_i}
        \right)^2 \diff x \\
        &\leq d \int_{\Omega} \sum_{i = 0}^{d} \sum_{j = 0}^{d} \left(
            \frac{\partial u_i}{\partial x_j}
        \right)^2 \diff x
        = d \int_{\Omega} \abs{\nabla u_h}^2 \diff x \\
        &= d \norm{\nabla u_h}_{L^2}^2.
    \end{align*}
    Using this, we have
    \begin{equation}
        b(q_h, u_h) \leq \sqrt{d} \norm{q_h}_{L^2} \norm{\nabla u_h}_{L^2},
    \end{equation}
    and as such \cref{eq:6.16} is satisfied.

    For \cref{eq:6.17}, we have
    \begin{equation}
        a(u_h, u_h)
        = \int_{\Omega} \nabla u_h : \nabla u_h \diff x
        = \norm{\nabla u_h}_{L^2}^2 = \abs{u_h}_{H^1}^2.
    \end{equation}
    As we have previously shown that the $H^1$ semi-norm is equivalent to the $H^1$ norm, \cref{eq:6.17} is satisfied.

    We have thus shown that the conditions in \cref{eq:6.15,eq:6.16,eq:6.17} are satisfied for $V_h = H^1_0$ and $Q_h = L^2$.
\end{solution}
\end{comment}

\begin{exercise}{6.2}
    Show that the conditions in \cref{eq:6.15,eq:6.16,eq:6.17} are satisfied for Taylor--Hood and Mini discretizations. % chktex 8
    (Note that Crouzeix--Raviart is non-conforming so it is more difficult to prove these conditions for this case.) % chktex 8
\end{exercise}

\begin{solution}{6.2}
    The Taylor--Hood elements are defined by the basis functions % chktex 8
    \begin{equation}
        \begin{split}
            u &: N_i = a_i + b_i x + c_i y + d_i x y + e_i x^2 + f_i y^2, \\
            p &: L_i = k_i + l_i x + m_i y.
        \end{split}
    \end{equation}
    We then have
    \begin{align*}
        \nabla u &=
        \begin{bmatrix}
            b_i + d_i y + 2 e_i x \\
            c_i + d_i x + 2 f_i y
        \end{bmatrix},
        &
        \nabla \cdot u &= ?
        &
        \nabla p &=
        \begin{bmatrix}
            l_i \\
            m_i
        \end{bmatrix}.
    \end{align*}

    \begin{equation}
        u_h = \sum_{i = 1}^{N} u_i N_i,
    \end{equation}
\end{solution}