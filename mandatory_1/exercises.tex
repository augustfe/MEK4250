\begin{exercise}{4.6}
    Consider the eigenvalues of the operators, $L_1$, $L_2$, and $L_3$, where $L_1 u = u_x$, $L_2 u = -\alpha u_{xx}$, $\alpha = 1.0 e^{-5}$, and $L_3 = L_1 + L_2$, with homogeneous Dirichlet conditions.
    For which of the operators are the eigenvalues positive and real?
    Repeat the exercise with $L_1 = x u_x$.
\end{exercise}

\begin{comment}
    \begin{solution}{4.6}
    For this exercise, I'm assuming that the domain is $[0, 1]$.
    For $L_1$, the eigenfunctions are of the from
    \begin{equation}
        u(x) = C e^{\lambda x},
    \end{equation}
    as we then have
    \begin{equation}
        L_1 C e^{\lambda x} = \frac{\partial}{\partial x} C e^{\lambda x} = C \lambda e^{\lambda x}.
    \end{equation}
    For the boundary conditions, we have $u(0) = u(1) = 0$, which gives us
    \begin{equation}
        C e^{\lambda \cdot 0} = C e^0 = C
    \end{equation}
    and
    \begin{equation}
        e^{\lambda \cdot 1} = C e^{\lambda},
    \end{equation}
    which gives us the trivial solution $C = 0$ and $u(x) = 0$, for all $\lambda$.

    For $L_2$, we consider first the operator $\frac{\partial^2}{\partial x^2}$.
    The eigenfunctions of this operator are loosely of the form
    \begin{equation}
        u(x) = e^{i x},
    \end{equation}
    as then
    \begin{equation}
        \frac{\partial^2}{\partial x^2} e^{i x} = -i^2 e^{i x} = -e^{i x}.
    \end{equation}
    In order to get the factor $\alpha$, we need to adjust the exponent such that we have
    \begin{equation}
        u(x) = e^{i \sqrt{\alpha} x}.
    \end{equation}
\end{solution}
\end{comment}

\begin{solution}{4.6}
    For this exercise, let the domain be $[0, 1]$ for simplicity.
    For the first part of the exercise, we are seeking solutions to
    \begin{equation}
        u_x = \lambda u,
    \end{equation}
    for which the solutions are
    \begin{equation}
        u(x) = C e^{\lambda x}.
    \end{equation}
    The boundary conditions are $u(0) = u(1) = 0$, which gives us that $C = 0$, and thus $u(x) = 0$ for all $\lambda$.

    For the second part of the exercise, we are seeking solutions to
    \begin{equation}
        -\alpha u_{xx} = \lambda u,
    \end{equation}
    or equivalently
    \begin{equation}
        -u_{xx} = \tfrac{\lambda}{\alpha} u.
    \end{equation}
    As the second derivative is a scaled version of the function itself, with opposite sign, we can guess that the solutions are of the form
    \begin{equation}
        u(x) = \sin(C x)
        \quad \text{or} \quad
        u(x) = \cos(C x).
    \end{equation}
    As the boundary conditions are homogeneous Dirichlet conditions, we can see that the sine function is the only one that satisfies the boundary conditions, for $C = k \pi$, $k = 0, 1, 2, \ldots$.
    Taking the double derivative of the sine function, we get
    \begin{equation}
        -u_{xx} = (k \pi)^2 \sin(k \pi x),
    \end{equation}
    and hence
    \begin{equation}
        \tfrac{\lambda}{\alpha} = (k \pi)^2 \implies \lambda = \alpha (k \pi)^2.
    \end{equation}
    As $\alpha > 0$, we see that the eigenvalues are positive and real for $L_2$, for $k \geq 1$.

    For $L_3$, we are looking for a solution
    \begin{equation}
        u_x - \alpha u_{xx} = \lambda u.
    \end{equation}
    Recalling back to videregående ODE solving, we can guess that the solution is of the form
    \begin{equation}
        u(x) = e^{rx},
    \end{equation}
    which gives the equation
    \begin{equation}
        -\alpha r^2 + r - \lambda = 0.
    \end{equation}
    Recalling the abc-formula, we then have the solutions
    \begin{align}
        r &=
        \frac{-1 \pm \sqrt{1 - 4 \alpha \lambda}}{-2 \alpha}
        = \frac{1 \mp \sqrt{1 - 4 \alpha \lambda}}{2 \alpha}.
    \end{align}
    For $1 - 4 \alpha \lambda > 0$, we have two real roots, however they only permit the trivial solution.
    For $1 - 4 \alpha \lambda = 0$, we have one real root, $r = 1 / 2\alpha$, which gives us a solution of the form
    \begin{equation}
        (C_1 + C_2 x) e^{x / 2\alpha},
    \end{equation}
    however for the boundary conditions, we must have $C_1 = C_2 = 0$, and hence the only solution is the trivial one.

    Finally, for $1 - 4 \alpha \lambda < 0$, we have two complex roots
    \begin{equation}
        r = \frac{1}{2\alpha} \pm i \frac{\sqrt{4 \alpha \lambda - 1}}{2\alpha},
    \end{equation}
    which gives us a solution of the form
    \begin{equation}
        e^{x / 2\alpha} \left(
            C_1 \cos\left(\frac{\sqrt{4 \alpha \lambda - 1}}{2\alpha} x\right)
            + C_2 \sin\left(\frac{\sqrt{4 \alpha \lambda - 1}}{2\alpha} x\right)
        \right).
    \end{equation}
    At $x = 0$, we have
    \begin{equation}
        e^{0} \left(
            C_1 \cos\left(\frac{\sqrt{4 \alpha \lambda - 1}}{2\alpha} \cdot 0 \right)
            + C_2 \sin\left(\frac{\sqrt{4 \alpha \lambda - 1}}{2\alpha} \cdot 0 \right)
        \right),
    \end{equation}
    such that we must have $C_1 = 0$.
    At $x = 1$, we have
    \begin{equation}
        e^{1 / 2\alpha} \left( C_2 \sin\left(\frac{\sqrt{4 \alpha \lambda - 1}}{2\alpha}\right) \right) = 0,
    \end{equation}
    which is valid when $\frac{\sqrt{4 \alpha \lambda - 1}}{2\alpha} = k \pi$, $k = 0, 1, 2, \ldots$.
    This gives us the eigenvalues
    \begin{align*}
        \frac{\sqrt{4 \alpha \lambda - 1}}{2\alpha} &= k \pi \\
        4 \alpha \lambda - 1 &= (2\alpha k \pi)^2 \\
        \lambda &= \frac{(2 \alpha k \pi)^2 + 1}{4 \alpha} \\
        \lambda &= \alpha (k \pi)^2 + \frac{1}{4 \alpha},
    \end{align*}
    which are always positive and real.

    Changing $L_1$ to $x u_x$, we have the equation
    \begin{equation}
        x u_x = \lambda u \implies u_x = \frac{\lambda}{x} u.
    \end{equation}
    As far as I can see, the monomials satisfy the equation, as
    \begin{equation}
        u_x = \frac{\partial}{\partial x} x^n = n x^{n - 1} = \frac{n}{x} x^n = \frac{n}{x} u.
    \end{equation}
    Looking even closer, we see that we do not even need to restrict ourselves the monomials, but can rather have
    \begin{equation}
        u(x) = C x^\lambda,
        \quad \text{for }
        \lambda \in \mathbb{R}.
    \end{equation}
    At $x = 0$ we have $u(0) = 0$, while we at $x = 1$ have
    \begin{equation}
        u(1) = C,
    \end{equation}
    and as such we sadly only have the trivial solution $u(x) = 0$ for all $\lambda$.

    I have no clue as to how to find the eigenvalues for $L_3$ when $L_1 = x u_x$.
\end{solution}