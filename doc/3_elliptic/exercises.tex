\section{The finite element method for elliptic problems}

\begin{exercise}
    Let $\Omega = (0, 1)$.
    Show that
    \begin{equation*}
        a(u, v) = \int_\Omega u \, v \ dx
    \end{equation*}
    is a bilinear form.
\end{exercise}

\begin{solution}
    To show that $a(u, v)$ is a bilinear form, we need to show that it is linear in both arguments.
    We can firstly note that
    \begin{equation*}
        a(u, v) = \int_\Omega u \, v \ dx = \int_\Omega v \, u \ dx = a(v, u),
    \end{equation*}
    showing that $a(u, v)$ is symmetric.
    We therefore only need to show that it is linear in one of the arguments.

    Let $u, v, w \in V$ and $\alpha, \beta \in \mathbb{R}$.
    Then
    \begin{align*}
        a(\alpha u + \beta v, w) &= \int_\Omega (\alpha u + \beta v) \, w \ dx \\
        &= \int_\Omega \alpha u \, w + \beta v \, w \ dx \\
        &= \alpha \int_\Omega u \, w \ dx + \beta \int_\Omega v \, w \ dx \\
        &= \alpha a(u, w) + \beta a(v, w),
    \end{align*}
    showing that $a(u, v)$ is linear in the first argument, and therefore a bilinear form.
\end{solution}

\begin{exercise}
    Let $\Omega = (0, 1)$.
    Show that
    \begin{equation*}
        a(u, v) = \int_\Omega u \, v \ dx
    \end{equation*}
    forms an inner product.
\end{exercise}

\begin{solution}
    To show that $a(u, v)$ forms an inner product, we need to show that it is symmetric, positive definite, and linear in the first argument.
    We have already shown that $a(u, v)$ is symmetric and linear in the previous exercise.
    We can also see that
    \begin{equation*}
        a(u, u) = \int_\Omega u \, u \ dx = \int_\Omega u^2 \ dx \geq 0,
    \end{equation*}
    showing that $a(u, u)$ is positive definite.
    We have therefore shown that $a(u, v)$ forms an inner product.
\end{solution}

\begin{exercise}
    Let $\Omega = (0, 1)$, then for all functions in $H^1_0(\Omega)$, Poincaré's inequality states that
    \begin{equation*}
        \lvert u \rvert_{L^2} \leq C \left\lvert \frac{\partial u}{\partial x} \right\rvert_{L^2}.
    \end{equation*}
    Use this inequality to show that the $H^1$ semi-norm defines a norm equivalent with the standard $H^1$ norm on $H^1_0(\Omega)$.
\end{exercise}