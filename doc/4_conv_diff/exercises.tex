\section{Discretization of a convection-diffusion problem}

\begin{exercise}
    Show that the matrix obtained from a central difference scheme applied to the operator $L u = u_x$ is skew-symmetric.
    Furthermore, show that the matrix obtained by linear continuous Lagrange elements are also skew-symmetric.
    Remark: The matrix is only skew-symmetric in the interior of the domain, not at the boundary.
\end{exercise}

\begin{solution}
    We consider the operator $L u = u_x$.
    The central difference scheme applied to this operator is
    \begin{equation*}
        L_h u = \frac{u_{i+1} - u_{i-1}}{2h},
    \end{equation*}
    where $h$ is the mesh size.
    The matrix representation of this operator is, considering the interior points only,
    \begin{equation*}
        L_h = \frac{1}{2h}
        \begin{bmatrix}
            -1 & 0 & 1 \\
            & -1 & 0 & 1 \\
            && \ddots & \ddots & \ddots \\
            &&& -1 & 0 & 1
        \end{bmatrix}.
    \end{equation*}
    This matrix is skew-symmetric, as can be seen by transposing it and negating it, ignoring the mismatched dimension of the matrix caused by ignoring the boundaries.

    The matrix $A$ for describing $L u = u_x$ using linear continuous Lagrange elements is defined by the elements
    \begin{equation*}
        A_{ij} = \int_{\Omega} \phi_i' \phi_j \, dx.
    \end{equation*}
    As the basis functions are linear, the derivative of the basis functions are constant.
    If there is no overlap between the basis functions, the integral is zero, and we clearly have $A_{ij} = 0 = -A_{ji}$.
    Suppose then that they do contain some overlap, on an interval $[x_l, x_u]$.
    Then we have
    \begin{equation*}
        A_{ij} = \int_{\Omega} \phi_i' \phi_j \, dx = \int_{x_l}^{x_u} \phi_i' \phi_j \, dx = -\int_{x_l}^{x_u} \phi_i \phi_j' \, dx + \left[ \phi_i \phi_j \right]_{x_l}^{x_u} = -A_{ji} + \left[ \phi_i \phi_j \right]_{x_l}^{x_u}.
    \end{equation*}
    Now, as the exercise hints to, for the interior points, the boundary term is zero, and we have $A_{ij} = -A_{ji}$.
\end{solution}

\begin{exercise}
    Estimate numerically the constant in Cea’s lemma for various $\alpha$ and $h$ for the Example 4.1.
\end{exercise}

\begin{solution}
    Cea's lemma states:
    \begin{theorem}[Cea's lemma]
        Suppose the conditions for Lax-Milgram's theorem are satisfied and that we solve the linear problem of finding $u_h \in V_{h, g}$ such that
        \begin{equation*}
            a(u_h, v_h) = L(v_h) \quad \forall v_h \in V_{h, 0}
        \end{equation*}
        on a finite element space of order $t$.
        Then,
        \begin{equation*}
            \norm{u - u_h}_V
            \leq
            C_1 h^t \norm{u}_{t+1}.
        \end{equation*}
        Here $C_1 = \frac{CB}{\alpha}$, where $B$ comes from the approximation property and $\alpha$ and $C$ are the constants of Lax-Milgram's theorem.
    \end{theorem}

    I'm unsure about how to bound the constant from above, but we can at least bound it from below by
    \begin{equation*}
        C_1 \geq \frac{\norm{u - u_h}_V}{\norm{u}_{t+1} h^t}.
    \end{equation*}
    In example 4.1 we consider the 1D convection diffusion problem, with $w = 1$, defined by
    \begin{align*}
        -u_x - \mu u_{xx} &= 0, \\
        u(0) &= 0, \\
        u(1) &= 1.
    \end{align*}
    The analytical solution to this is
    \begin{equation*}
        u(x) = \frac{e^{-x / \mu} - 1}{e^{-1 / \mu} - 1}.
    \end{equation*}

    This leads to the following estimates for $C_1$ for varying $\mu$ and $h$, as shown in Table~\ref{tab:cea}.

    \begin{table}[h]
        \centering
        \caption{Estimates of $C_1$ for various $\mu$ and $h$.\label{tab:cea}}
        \begin{tabular}{lrrr}
            \toprule
            $h \backslash \mu$ & 0.100 & 0.010 & 0.001 \\
            \midrule
            0.1000 & 0.525379 & 15.970240 & 149.093342 \\
            0.0100 & 0.054700 & 5.622143 & 162.507563 \\
            0.0010 & 0.005472 & 0.583342 & 56.752388 \\
            0.0001 & 0.000547 & 0.058356 & 5.884473 \\
            \bottomrule
        \end{tabular}
    \end{table}
\end{solution}