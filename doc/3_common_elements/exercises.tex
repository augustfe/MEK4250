\section{Some common finite elements}

\begin{exercise}
    Consider two triangles $T_{0}$ and $T_{1}$ defined by the vertices $(0, 0)$, $(1, 0)$, $(0, 1)$ and $(3, 4)$, $(4, 2)$, $(4, 4)$.
    Compute the mapping between them.
\end{exercise}

\begin{solution}
    Let $x$ and $\hat{x}$ be the coordinates in the reference element $T_{0}$ and the target element $T_{1}$, respectively.
    We seek then to find the mappings
    \begin{align*}
        x &= F_{T_{0}}(\hat{x}) = A_{T_{0}} \hat{x} + x_0, \\
        \hat{x} &= F_{T_{1}}(x) = A_{T_{1}} x + \hat{x}_0,
    \end{align*}
    The Jacobian of the mapping is
    \begin{equation}
        \frac{\partial x}{\partial \hat{x}} = J(\hat{x}) = A_{T_{0}}.
    \end{equation}
    Enumerating the vertices, we have
    \begin{align*}
        x_0 &= (0, 0) & x_1 &= (1, 0) & x_2 &= (0, 1) \\
        \hat{x}_0 &= (3, 4) & \hat{x}_1 &= (4, 2) & \hat{x}_2 &= (4, 4).
    \end{align*}
    We have that
    \begin{equation*}
        F_{T_1}(x_0) = A_{T_1}
        \begin{bmatrix}
            0 \\
            0
        \end{bmatrix}
        + \hat{x}_0 = \hat{x}_0
    \end{equation*}
    Letting $A_{T_1} =
    \begin{bmatrix}
        a & b \\
        c & d
    \end{bmatrix}$, we have
    \begin{align*}
        \hat{x}_1 &=
        \begin{bmatrix}
            a & b \\
            c & d
        \end{bmatrix}
        \begin{bmatrix}
            1 \\
            0
        \end{bmatrix} + \hat{x}_0 \\
        \begin{bmatrix}
            4 \\
            2
        \end{bmatrix}
        &=
        \begin{bmatrix}
            a & b \\
            c & d
        \end{bmatrix}
        \begin{bmatrix}
            1 \\
            0
        \end{bmatrix} +
        \begin{bmatrix}
            3 \\
            4
        \end{bmatrix} \\
        \begin{bmatrix}
            1 \\
            -2
        \end{bmatrix}
        &=
        \begin{bmatrix}
            a \\ c
        \end{bmatrix}
    \end{align*}
    Similarly, we find that $(b, c) = \hat{x}_2 - \hat{x}_0 = (1, 0)$, such that
    \begin{equation*}
        A_{T_1} =
        \begin{bmatrix}
            1 & 1 \\
            -2 & 0
        \end{bmatrix}
    \end{equation*}
    Then,
    \begin{align*}
        \hat{x} &= A_{T_1} x + \hat{x}_0 \\
        A_{T_1} x &= \hat{x} - \hat{x}_0 \\
        x &= A_{T_1}^{-1} (\hat{x} - \hat{x}_0) \\
        x &= A_{T_1}^{-1} \hat{x} - A_{T_1}^{-1} \hat{x}_0 \\
        x &= A_{T_0} \hat{x} + x_0
    \end{align*}
    Then, we need only invert $A_{T_1}$.
    \begin{align*}
        \begin{bmatrix}
            1 & 1 \\
            -2 & 0
        \end{bmatrix} \Big\vert
        \begin{bmatrix}
            1 & 0 \\
            0 & 1
        \end{bmatrix}
        &\rightarrow
        \begin{bmatrix}
            1 & 1 \\
            0 & 2
        \end{bmatrix} \Big\vert
        \begin{bmatrix}
            1 & 0 \\
            2 & 1
        \end{bmatrix}
        \rightarrow
        \begin{bmatrix}
            1 & 1 \\
            0 & 1
        \end{bmatrix} \Big\vert
        \begin{bmatrix}
            1 & 0 \\
            1 & \frac{1}{2}
        \end{bmatrix}
        \rightarrow
        \begin{bmatrix}
            1 & 0 \\
            0 & 1
        \end{bmatrix} \Big\vert
        \begin{bmatrix}
            0 & -\frac{1}{2} \\
            1 & \frac{1}{2}
        \end{bmatrix},
    \end{align*}
    such that the inverse mapping is
    \begin{equation*}
        A_{T_0} = \frac{1}{2}
        \begin{bmatrix}
            0 & -1 \\
            2 & 1
        \end{bmatrix}
    \end{equation*}
\end{solution}

\begin{exercise}
    Make a Python code that defines a Hermite element on the unit interval.
\end{exercise}

\begin{solution}
    We consider Hermite interpolation onto $k + 1$ points, with a single derivative at each node.
    That is, they satisfy
    \begin{equation}
        \begin{split}
            d_i(H_j) &= H_j(x_i) = \delta_{ij}, \quad \text{for even } i, j, \\
            d_i(H_j) &= H_j'(x_i) = \delta_{ij}, \quad \text{for odd } i, j.
        \end{split}
    \end{equation}
\end{solution}

\begin{exercise}
    Make a Python code that defines a Lagrange element of arbitrary order on the reference triangle consisting of the vertices $(0, 0)$, $(1, 0)$, and $(0, 1)$.
    Let $\mathbb{P}_k = \{x^i y^j\}$ for $i,j$ such that $i + j \leq k$.
\end{exercise}

\begin{solution}
    In order to illustrate the goal, the nodes of the first four Lagrange elements are shown in \cref{fig:lagrange_elements_duplicate}.
    We firstly need to decide how we are going to compute the polynomials.
    The easiest method is perhaps to set up the Vandermonde matrix
    \begin{equation*}
        \begin{bmatrix}
            1 & x_0 & y_0 & x_0^2 & x_0 y_0 & y_0^2 & \ldots \\
            1 & x_1 & y_1 & x_1^2 & x_1 y_1 & y_1^2 & \ldots \\
            1 & x_2 & y_2 & x_2^2 & x_2 y_2 & y_2^2 & \ldots \\
            \vdots & \vdots & \vdots & \vdots & \vdots & \vdots & \ddots
        \end{bmatrix},
    \end{equation*}
    that is, with all the monomials of the form $x^i y^j$ for $i + j \leq k$, for each of the nodes.
    Inverting this matrix gives us the coefficients of the Lagrange polynomials.

    \begin{figure}[!h]
        \centering
        \def\NTri{4}
\begin{tikzpicture}[scale=2]
    \foreach \n in {1, ..., \NTri} {
        \begin{scope}[xshift=\n*2 cm]

            \coordinate (A) at (0, 1);
            \coordinate (B) at (0, 0);
            \coordinate (C) at (1, 0);

            \draw[myblue, ultra thick] (A) -- (B) -- (C) -- cycle;

            \foreach \i in {0, ..., \n} {
                \foreach \j in {0, ..., \i} {
                    % Draw a point for each Lagrange node
                    \coordinate (P) at ($(A) - \i/(\n)*(A) + \j/(\n)*(C)$);
                    \fill[mypurple] (P) circle (1.2pt);
                }
            }
        \end{scope}
    }
\end{tikzpicture}
        \caption{Lagrange elements of order up to 4 on the reference triangle.\label{fig:lagrange_elements_duplicate}}
    \end{figure}

    We would however generally want to avoid inverting matrices, as this is both costly, and can lead to numerical instability.
    Instead, we can use a definition of the Lagrange polynomials on triangles, analogously to the one-dimensional case.
    This process is illustrated in \cref{fig:lagrange_elements_2}.
    The trick is to use the equations for lines in order to zero out the other nodes, in the same way as we use the points directly in the one-dimensional case.

    We denote the points on the triangle by a triplet $(i, j, k)$, in reference to the corners, in a manor such that $i + j + k = n$, where $n$ is the order of the polynomial.
    In the example, we wish to find the Lagrange polynomial which is equal to $1$ at the point $(\frac{1}{4}, \frac{1}{4})$, which corresponds to the triplet $(2, 1, 1)$.
    This already contains the information about which lines we need, namely the two diagonal lines above, the one vertical line to the left, and the one horizontal line below.
    These lines are marked with a thick line in the figure.

    Multiplying the equations for these lines together gives us the polynomial
    \begin{equation*}
        p(x, y) = (x + y - 1)\left(x + y - \tfrac{3}{4}\right)xy,
    \end{equation*}
    which indeed is zero at all other nodes.
    It does not however equal $1$ at our desired point, which is easily fixed by normalizing the value, giving
    \begin{equation*}
        \ell_{211}(x, y) = \frac{
            (x + y - 1)\left(x + y - \tfrac{3}{4}\right)xy
        }{
            \left(\frac{1}{4} + \frac{1}{4} - 1\right)\left(\frac{1}{4} + \frac{1}{4} - \frac{3}{4}\right)\left(\frac{1}{4}\right)^2
        }
        = 128xy(x + y - 1)\left(x + y - \tfrac{3}{4}\right).
    \end{equation*}

    \begin{figure}[!h]
        \centering
        \begin{tikzpicture}[scale=7]
    % Draw horizontal lines
    \foreach \i in {0, ..., 4} {
        % \draw[myblue, ultra thick] (A) -- ($(\i/4, 0)$);
        \draw[myblue, ultra thick, dashed] ($(0, \i/4)$) -- ($(1 - \i/4, \i/4)$);
    }

    % Draw vertical lines
    \foreach \i in {0, ..., 4} {
        % \draw[myblue, ultra thick] (A) -- ($(\i/4, 0)$);
        \draw[myblue, ultra thick, dashed] ($(\i/4, 0)$) -- ($(\i/4, 1 - \i/4)$);
    }

    % Draw diagonal lines
    \foreach \i in {0, ..., 4} {
        % \draw[myblue, ultra thick] (A) -- ($(\i/4, 0)$);
        \draw[myblue, ultra thick, dashed] ($(0, 1 - \i/4)$) -- ($(1 - \i/4, 0)$);
    }

    \draw[ultra thick] (1, 0) -- (0, 1);
    \draw[ultra thick] (3/4, 0) -- (0, 3/4);
    \draw[ultra thick] (0, 0) -- (0, 1);
    \draw[ultra thick] (0, 0) -- (1, 0);

    \fill[red] (1/4, 1/4) circle (0.5pt);

    \foreach \i in {0, ..., \the\numexpr 3 + 1\relax} {
        \foreach \j in {0, ..., \the\numexpr 3 + 1 - \i\relax} {
            \node[above right] at (\i/4, \j/4) {$\the\numexpr 4 - \i - \j\relax\i\j$};
        }
    }
\end{tikzpicture}
        \caption{Lagrange elements of order up to 4 on the reference triangle.\label{fig:lagrange_elements_2}}
    \end{figure}

    This process is implemented in \verb|lagrange.py|.
\end{solution}
