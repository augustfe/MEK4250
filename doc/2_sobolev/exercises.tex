\section{Crash course in Sobolev Spaces}

\begin{exercise}
    What is a norm?
    Show that
    \begin{equation*}
        \lVert u \rVert_p = \left( \int_{0}^{1} \lvert u \rvert^p \ dx \right)^{1/p}
    \end{equation*}
    defines a norm.
\end{exercise}

\begin{solution}
    Following the definition in Spaces by Tom Lindstrøm, a norm is a function $\lVert \cdot \rVert: V \to \mathbb{R}$, where $V$ is a vector space, such that
    \begin{enumerate}[label=(\roman*)] % chktex 36
        \item $\lVert \boldsymbol{u} \rVert \geq 0$ with equality if and only if $\boldsymbol{u} = \boldsymbol{0}$.

        \item $\lVert \alpha \boldsymbol{u} \rVert = \lvert \alpha \rvert \lVert \boldsymbol{u} \rVert$ for all $\alpha \in \mathbb{R}$ and all $\boldsymbol{u} \in V$.

        \item (Triangle Inequality for Norms) $\lVert \boldsymbol{u} + \boldsymbol{v} \rVert \leq \lVert \boldsymbol{u} \rVert + \lVert \boldsymbol{v} \rVert$ for all $\boldsymbol{u}, \boldsymbol{v} \in V$.
    \end{enumerate}

    Positivity is clear, as $\lvert u \rvert^p \geq 0$ for all $u \in L^p(0, 1)$.
    The only way $\lVert u \rVert_p = 0$ is if $\lvert u \rvert^p = 0$.
    Homogeneity is also clear, as
    \begin{align*}
        \lVert \alpha u \rVert_p &= \left( \int_{0}^{1} \lvert \alpha u \rvert^p \ dx \right)^{1/p} \\
        &= \left( \int_{0}^{1} \lvert \alpha \rvert^p \lvert u \rvert^p \ dx \right)^{1/p} \\
        &= \lvert \alpha \rvert \left( \int_{0}^{1} \lvert u \rvert^p \ dx \right)^{1/p} \\
        &= \lvert \alpha \rvert \lVert u \rVert_p.
    \end{align*}
    The triangle inequality is a bit more involved, but we have
    \begin{align*}
        \lVert u + v \rVert_p^p &= \int_{0}^{1} \lvert u + v \rvert^p \ dx \\
        &\leq \int_{0}^{1} \left( \lvert u \rvert + \lvert v \rvert \right)^p \ dx \\
        &\leq \int_{0}^{1} \lvert u \rvert^p + \lvert v \rvert^p \ dx \\
        &= \lVert u \rVert_p^p + \lVert v \rVert_p^p,
    \end{align*}
    which implies
    \begin{equation*}
        \lVert u + v \rVert_p \leq \left( \lVert u \rVert_p^p + \lVert u \rVert_p^p \right)^{1/p} \leq \lVert u \rVert_p + \lVert u \rVert_p.
    \end{equation*}
\end{solution}

\begin{exercise}
    What is an inner product? Show that
    \begin{equation*}
        (u, v)_k = \sum_{i \leq k} \int_{\Omega} \left( \frac{\partial u}{\partial x} \right)^i \left( \frac{\partial v}{\partial x} \right)^i \ dx
    \end{equation*}
    defines an inner product.
\end{exercise}

\begin{solution}
    Again, Spaces by Tom Lindstrøm defines an inner product as a function $(\cdot, \cdot): V \times V \to \mathbb{R}$, where $V$ is a vector space, such that
    \begin{enumerate}[label=(\roman*)] % chktex 36
        \item $(\boldsymbol{u}, \boldsymbol{v}) = (\boldsymbol{v}, \boldsymbol{u})$ for all $\boldsymbol{u}, \boldsymbol{v} \in V$.

        \item $(\boldsymbol{u} + \boldsymbol{v}, \boldsymbol{w}) = (\boldsymbol{u}, \boldsymbol{w}) + (\boldsymbol{v}, \boldsymbol{w})$ for all $\boldsymbol{u}, \boldsymbol{v}, \boldsymbol{w} \in V$.

        \item $(\alpha \boldsymbol{u}, \boldsymbol{v}) = \alpha (\boldsymbol{u}, \boldsymbol{v})$ for all $\alpha \in \mathbb{R}$, $\boldsymbol{u}, \boldsymbol{v} \in V$.

        \item For all $\boldsymbol{u} \in V$, $(\boldsymbol{u}, \boldsymbol{u}) \geq 0$ with equality if and only if $\boldsymbol{u} = \boldsymbol{0}$.
    \end{enumerate}

    Symmetry is clear, as
    \begin{equation*}
        (u, v)_k
        = \sum_{i \leq k} \int_{\Omega} \left( \frac{\partial u}{\partial x} \right)^i \left( \frac{\partial v}{\partial x} \right)^i \ dx
        = \sum_{i \leq k} \int_{\Omega} \left( \frac{\partial v}{\partial x} \right)^i \left( \frac{\partial u}{\partial x} \right)^i \ dx
        = (v, u)_k.
    \end{equation*}
    Linearity in the first argument is also satisfied, as
    \begin{align*}
        (u + v, w)_k
        &= \sum_{i \leq k} \int_{\Omega} \left( \frac{\partial (u + v)}{\partial x} \right)^i \left( \frac{\partial w}{\partial x} \right)^i \ dx \\
        &= \sum_{i \leq k} \int_{\Omega} \left( \frac{\partial u}{\partial x} + \frac{\partial v}{\partial x} \right)^i \left( \frac{\partial w}{\partial x} \right)^i \ dx \\
        &= \sum_{i \leq k} \int_{\Omega} \left( \frac{\partial u}{\partial x} \right)^i \left( \frac{\partial w}{\partial x} \right)^i + \left( \frac{\partial v}{\partial x} \right)^i \left( \frac{\partial w}{\partial x} \right)^i \ dx \\
        &= \sum_{i \leq k} \int_{\Omega} \left( \frac{\partial u}{\partial x} \right)^i \left( \frac{\partial w}{\partial x} \right)^i \ dx + \sum_{i \leq k} \int_{\Omega} \left( \frac{\partial v}{\partial x} \right)^i \left( \frac{\partial w}{\partial x} \right)^i \ dx \\
        &= (u, w)_k + (v, w)_k.
    \end{align*}
    Homogeneity in the first argument is also satisfied, as
    \begin{align*}
        (\alpha u, v)_k
        &= \sum_{i \leq k} \int_{\Omega} \left( \frac{\partial (\alpha u)}{\partial x} \right)^i \left( \frac{\partial v}{\partial x} \right)^i \ dx \\
        &= \sum_{i \leq k} \int_{\Omega} \alpha \left( \frac{\partial u}{\partial x} \right)^i \left( \frac{\partial v}{\partial x} \right)^i \ dx \\
        &= \alpha \sum_{i \leq k} \int_{\Omega} \left( \frac{\partial u}{\partial x} \right)^i \left( \frac{\partial v}{\partial x} \right)^i \ dx \\
        &= \alpha (u, v)_k.
    \end{align*}
    Finally, positivity is also satisfied, as
    \begin{equation*}
        (u, u)_k
        = \sum_{i \leq k} \int_{\Omega} \left( \frac{\partial u}{\partial x} \right)^i \left( \frac{\partial u}{\partial x} \right)^i \ dx
        = \sum_{i \leq k} \int_{\Omega} \left( \frac{\partial^i u}{\partial x^i} \right)^{2} \ dx
        \geq 0.
    \end{equation*}
    $(u, u)_k = 0$ only if $\frac{\partial^i u}{\partial x^i} = 0$ for all $i \leq k$, which implies $u = 0$.
    $(u, v)_k$ is therefore an inner product.
\end{solution}