\section{The finite element method}
Explain Ciarlet's definition of a finite element.
Explain the concept of functionals and function spaces.
How are degrees of freedom used to ensure that the finite element spaces are part of certain function spcaes?
Show that a finite element may conveniently be defined in terms of a reference element.
List common elements in common spaces.

\subsection{Ciarlet's definition of a finite element}
Ciarlet defines a finite element by a triplet $(T, V , D)$, where
\begin{itemize}
    \item
        $T$ is a bounded domain in $R^d$, most typically a polyhedron.

    \item
        $V = \{\psi_i\}_{i = 1}^n$ is a set of linearly independent basis functions on $T$

    \item
        $D = \{d_i\}_{i = 1}^n$ is a set of (linearly independent) degrees of freedom defined in terms of linear functionals on $V$.
        (We remark for $v \in V$ we may evaluate $d_i(v)$ since $d_i$ is a linear functional on $V$.)
\end{itemize}
$T$ defines the triangulation, or cells, in our domain.
The basis functions are then just defined on their cell $T$.
The magic happens when we introduce $D$, dubbed the degrees of freedom, or dofs for short.
They are in a sense how we ``tie'' the function spaces together, as they are originally just defined locally.

Most elements are implemented through a nodal basis, defined by the set of basis functions $\{\phi_i\}_{i = 0}^n$ satsifying $d_j(\phi_i) = \delta_{ij}$.
The simplest example is the Lagrange element, where for a basis function $L_j$, the dofs are defined by
\begin{equation}
    d_i(L_j) = L_j(x_i) = \delta_{ij}.
\end{equation}
Initially, the set $\{x_i\}$ consists of the nodes of each triangle.

Say we have two triangles, each with the vertices $\{x_1, x_2, x_3\}$ and $\{x_4, x_5, x_6\}$ respectively.
If the triangles are next to eachother, we would perhaps then have that $x_2 = x_4$ and $x_3 = x_5$.
However, we still av the set of dofs $\{d_i\}_{i = 1}^{6}$, even though we only have four unique points.
This results in discontinuous Lagrange elements, as they do not directly communicate across the boundaries.
If we however say that $d_2 = d_4$ and $d_3 = d_5$, we would have continuous elements.

\subsection{Concept of functionals and function spaces}
A function space is a vector space where each element is a function.
Typically, the function spaces are defined by some properties, for instance the space of all continuous functions, all functions with a finite integral, and especially in our case all functions where the function as well as the derivatives have finite integrals.
This essentially forms our ``main'' space $H^1(\Omega)$, defined by
\begin{equation}
    H^1(\Omega) =
    \left\{
        f
        :
        \int_{\Omega} \abs{f}^2 + \abs{\nabla f}^2 \diff x < \infty
    \right\}.
\end{equation}
A functional is then simply something which takes in a vector, for instance from a function space, and returns a number, either real or complex (although typically real in our case).

\subsection{How DOFs define our space}
