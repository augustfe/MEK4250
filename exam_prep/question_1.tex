\section{Weak formulation and finite element error estimation}
\subsection*{Problem description}
Formulate a finite element method for the Poisson problem with a variable coefficient $\kappa : \Omega \to \mathbb{R}^{d \times d}$.
Assume that $\kappa$ is positive and symmetric.
Show that Lax--Milgram's theorem is satisfied. % chktex 8
Consider extensions to e.g.\ convection-diffusion equation and the elasticity equation.
Derive \textit{a priori} error estimates in terms of Cea's lemma for the finite element method in the energy norm.
Describe how to perform an estimation of convergence rates.

\subsection{Weak form of the Poisson equation}
The Poisson problem with a variable coefficient $\kappa$ is given by
\begin{equation}
    \begin{split}
        -\nabla \cdot (\kappa \nabla u) &= f \quad \text{in } \Omega, \\
        u &= g \quad \text{on } \partial\Omega_D, \\
        \kappa \frac{\partial u}{\partial n} &= h \quad \text{on } \partial\Omega_N,
    \end{split}
\end{equation}
with $\partial\Omega_D$ and $\partial\Omega_N$ disjoint parts of the boundary $\partial\Omega$.
Here, $\partial\Omega_D$ denotes the Dirichlet boundary, while $\partial\Omega_N$ denotes the Neumann boundary.

Setting up the weak formulation roughly follows the following steps:
\begin{enumerate}
    \item Multiply with a test function $v$ and integrate over the domain $\Omega$

    \item Integrate by parts, and apply Green's lemma.

    \item Apply the boundary conditions.
\end{enumerate}
Multiplying with a test function $v$ and integrating over the domain $\Omega$ gives us
\begin{equation}
    \int_\Omega -\nabla \cdot (\kappa \nabla u) \, v \, \diff x = \int_\Omega f \, v \, \diff x.
\end{equation}
This is however not ideal, as we are now required to have $u \in H^2(\Omega)$, which is not ideal.
We therefore apply Green's lemma to the left-hand side, which gives us
\begin{equation}
    \int_\Omega -\nabla \cdot (\kappa \nabla u) \, v \, \diff x
    = \int_\Omega \kappa \nabla u \cdot \nabla v \, \diff x - \int_{\partial\Omega} \kappa \frac{\partial u}{\partial n} \, v \, \diff s.
\end{equation}
This eases the requirements on $u$, as we now only require $u \in H^1(\Omega)$, while strengthening the requirements on $v$ to $v \in H^1(\Omega)$.

We can now apply the boundary conditions.
Splitting the boundary integral into two parts, we have
\begin{equation}
    \int_{\partial\Omega} \kappa \frac{\partial u}{\partial n} \, v \, \diff s = \int_{\partial\Omega_D} \kappa \frac{\partial u}{\partial n} \, v \, \diff s + \int_{\partial\Omega_N} \kappa \frac{\partial u}{\partial n} \, v \, \diff s.
\end{equation}
As we have a section of Dirichlet boundary, we need not solve for $u$ here, as we know the value of $u$ on this section.
We may therefore set $v = 0$ on $\partial\Omega_D$ by having $v \in H_0^1(\Omega)$, which gives us
\begin{equation}
    \int_{\partial\Omega_D} \kappa \frac{\partial u}{\partial n} \, v \, \diff s + \int_{\partial\Omega_N} \kappa \frac{\partial u}{\partial n} \, v \, \diff s
    = \int_{\partial\Omega_N} h \, v \, \diff s.
\end{equation}

This gives us the weak formulation for the Poisson problem
\begin{equation}
    \int_\Omega \kappa \nabla u \cdot \nabla v \, \diff x = \int_\Omega f \, v \, \diff x + \int_{\partial\Omega_N} h \, v \, \diff s.
\end{equation}

\subsection{Lax--Milgram's theorem} % chktex 8
Lax--Milgram's theorem states: % chktex 8
\begin{theorem}
    Let $V$ be a Hilbert space, $a(\cdot, \cdot)$ be a bilinear form, $L(\cdot)$ be a linear form, and let the following three conditions be satisfied:
    \begin{enumerate}
        \item $a(u, u) \geq \alpha \norm{u}_V^2$ for all $u \in V$, where $\alpha > 0$ is a constant.

        \item $a(u, v) \leq C \norm{u}_V \norm{v}_V$ for all $u, v \in V$, where $C > 0$ is a constant.

        \item $L(v) \leq D \norm{v}_V$ for all $v \in V$, where $D > 0$ is a constant.
    \end{enumerate}
    Then, the problem of finding $u \in V$ such that
    \begin{equation}
        a(u, v) = L(v) \quad \forall v \in V
    \end{equation}
    is well-posed in the sense that there exists a unique solution with the stability condition
    \begin{equation}
        \norm{u}_V \leq \frac{C}{\alpha} \norm{L}_{V^*}.
    \end{equation}
\end{theorem}

We can now show that Lax--Milgram's theorem is satisfied for the weak formulation of the Poisson problem. % chktex 8
We have the bilinear form
\begin{equation}
    a(u, v) = \int_\Omega \kappa \nabla u \cdot \nabla v \, \diff x,
\end{equation}
and the linear form
\begin{equation}
    L(v) = \int_\Omega f \, v \, \diff x + \int_{\partial\Omega_N} h \, v \, \diff s.
\end{equation}
We can now show that the three conditions of Lax--Milgram's theorem are satisfied. % chktex 8

Firstly we have that
\begin{align*}
    a(u, u)
    &= \int_\Omega \kappa \nabla u \cdot \nabla u \, \diff x
    = \int_\Omega (\nabla u)^T \kappa^T \nabla u \, \diff x \\
    &= \int_\Omega (\nabla u)^T \kappa \nabla u \, \diff x
    \geq \int_\Omega k_0 \abs{\nabla u}^2 \, \diff x \\
    &= k_0 \abs{u}^2_1
    \geq \alpha \norm{u}_1^2
\end{align*}
where we've first used the fact that $\kappa$ is symmetric, and then that $\kappa$ is positive, and finally that the $H^1_0$ semi-norm is equivalent to the $H^1_0$ norm.

For the second point, we firstly show that $a(\cdot, \cdot)$ defines an inner product, assuming that $\kappa$ is bounded.
The inequality then follows simply from the Cauchy--Schwarz inequality. % chktex 8
In order to show that $a(\cdot, \cdot)$ is symmetric, we simply have
\begin{equation}
    a(u, v) = \int_\Omega \kappa \nabla u \cdot \nabla v \, \diff x
    = \int_\Omega (\nabla v)^T \kappa \nabla u \, \diff x
    = a(v, u).
\end{equation}
We can now show that $a(u, v) \leq C \norm{u}_V \norm{v}_V$.
As $\kappa$ is bounded, we have
\begin{equation}
    \kappa \xi \cdot \xi \leq k_1 \abs{\xi}^2
\end{equation}
for all $\xi \in \mathbb{R}^d$.
This gives us that
\begin{equation}
    a(u, u) = \int_\Omega \kappa \nabla u \cdot \nabla u \, \diff x
    \leq k_1 \int_\Omega \abs{\nabla u}^2 \, \diff x
    \leq k_1 \abs{u}_1^2.
\end{equation}
We then have that
\begin{gather*}
    a(u, v)^2 \leq a(u, u) a(v, v) = k_1^2 \abs{u}_1^2 \abs{v}_1^2 \\
    a(u, v) \leq k_1 \abs{u}_1 \abs{v}_1
\end{gather*}
Additionally, for $H^1_0$ we have
\begin{align*}
    \norm{u}_1^2 = \norm{u}_{L^2}^2 + \abs{u}_1^2
    \leq C \abs{u}_1^2 + \abs{u}_1^2
    = (1 + C) \abs{u}_1^2
\end{align*}
by Poincaré's lemma, which gives us
\begin{equation}
    a(u, v) \leq k_1 \abs{u}_1 \abs{v}_1
    \leq \frac{k_1}{1 + C} \norm{u}_1 \norm{v}_1.
\end{equation}

Lastly, we have
\begin{align*}
    L(v)
    &= \int_\Omega f \, v \, \diff x + \int_{\partial\Omega_N} h \, v \, \diff s
    \leq \norm{f}_{L^2(\Omega)} \norm{v}_{L^2(\Omega)} + \norm{h}_{L^2(\partial\Omega_N)} \norm{v}_{L^2(\partial\Omega_N)} \\
    &\leq (\norm{f}_{L^2}(\Omega) + \norm{h}_{L^2(\partial\Omega_N)}) \norm{v}_{L^2(\Omega)}
    \leq D \norm{v}_{L^2(\Omega)}
    \leq D \norm{v}_1,
\end{align*}
showing that the third condition is satisfied as well.
Here, we assume that $f \in L^2(\Omega)$ and $h \in L^2(\partial\Omega_N)$, which is a reasonable assumption.

\subsection{Extension to convection-diffusion equation}
The convection-diffusion equation is given by
\begin{equation}
    \begin{split}
        -\nabla \cdot (\kappa \nabla u) + w \cdot \nabla u &= f \quad \text{in } \Omega, \\
        u &= g \quad \text{on } \partial\Omega_D, \\
        \kappa \frac{\partial u}{\partial n} &= h \quad \text{on } \partial\Omega_N,
    \end{split}
\end{equation}
where $w$ is the convection term.

{\Large To be continued}

\subsection{Cea's lemma}
Cea's lemma states that if $u_h$ is the finite element solution, and $u$ is the exact solution, then
\begin{equation}
    \norm{u - u_h}_V \leq \frac{C B}{\alpha} h^t \norm{u}_{t + 1},
\end{equation}
where $B$ is the constant derived from the polynomial approximation, $h$ is a measure of the mesh size, and $C$ and $\alpha$ are constants from Lax--Milgram's theorem. % chktex 8

Here, we assume that the energy norm is given by
\begin{equation}
    \norm{w}_E = a(w, w)^{1/2},
\end{equation}
and that we should find an error estimate in this norm.
We then have
\begin{align*}
    \norm{u - u_h}_E^2 &= a(u - u_h, u - u_h) \\
    &= a(u - u_h, u - v + v - u_h) \qquad v \in V \\
    &= a(u - u_h, u - v) + \underbrace{a(u - u_h, v - u_h)}_{0 \text{ as } v - u_h \in V} \\
    &= a(u - u_h, u - v) \\
    &\leq \norm{u - u_h}_E \norm{u - v}_E.
\end{align*}
Dividing by $\norm{u - u_h}_E$ gives us
\begin{equation}
    \norm{u - u_h}_E \leq \norm{u - v}_E.
\end{equation}
We then further have, choosing $v = I_h u$ as the polynomial approximation of order $t$,
\begin{align*}
    \norm{u - I_h u}_E^2 &= a(u - I_h u, u - I_h u) \\
    &\leq \frac{k_1}{1 + C} \norm{u - I_h u}_1^2 \\
    &\leq \frac{k_1}{1 + C} (B h^t)^2 \norm{u}_{t + 1}^2.
\end{align*}
This finally gives us that
\begin{equation}
    \norm{u - u_h}_E \leq \sqrt{\frac{k_1}{1 + C}} B h^t \norm{u}_{t + 1}.
\end{equation}
I believe this is what the question is asking for, but I am not sure.

\subsection{Convergence rates}
We can estimate the convergence rates by looking at the error estimates for varying mesh sizes $h$.
Considering the error estimates for $h = h_1$ and $h = h_2$, we have
\begin{equation*}
    \norm{u - u_{h_1}}_E \leq C h_1^t \norm{u}_{t + 1}
    \quad\text{and}\quad
    \norm{u - u_{h_2}}_E \leq C h_2^t \norm{u}_{t + 1}.
\end{equation*}
This gives
\begin{align*}
    \frac{\norm{u - u_{h_2}}_E}{\norm{u - u_{h_1}}_E}
    &\leq \frac{C h_2^t \norm{u}_{t + 1}}{C h_1^t \norm{u}_{t + 1}} \\
    &= \left(\frac{h_2}{h_1}\right)^t.
\end{align*}
Taking the logarithm of both sides gives us
\begin{align*}
    \log \left(\frac{\norm{u - u_{h_2}}_E}{\norm{u - u_{h_1}}_E}\right)
    &\leq t \log \left(\frac{h_2}{h_1}\right) \\
    \frac{
        \log \left(\frac{\norm{u - u_{h_2}}_E}{\norm{u - u_{h_1}}_E}\right)
    }{
        \log \left(\frac{h_2}{h_1}\right)
    }
    &\leq t.
\end{align*}
If we solve for varying mesh sizes $\{h_i\}_{i=1}^n$, keeping $h_{i+1} / h_i$ constant, we get a series of lower bounds for the convergence rate $t$.