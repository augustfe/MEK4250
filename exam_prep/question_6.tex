\section{Other formulations}
Consider the various formulations of the Poisson problem.
Show that minimization of energy
\begin{equation}
    \int_\Omega \frac{1}{2}(\nabla u)^2 - f \, u \diff x
\end{equation}
corresponds to a weak formulation that through Green's lemma gives a strong formulation.
Do the same thing for the least square formulation and arrive at a bi-harmonic equation with some additional boundary conditions.
Consider both the mixed formulation and least square formulation of the mixed formulation, and note that only the first requires the Brezzi conditions.

\subsection{Short-form answer}
The strong formulation of the Poisson problem is
\begin{equation}
    -\Delta u_S = f \quad \text{in } \Omega,
    \quad u_S = g \quad \text{on } \partial\Omega.
\end{equation}
This formulation requires that $u_S \in C^2(\Omega)$, given an $f \in C^0(\Omega)$.
The weak formulation on the other hand is to find $u_W \in H^1_g(\Omega)$ given an $f \in L^2(\Omega)$ such that
\begin{equation}
    \int_\Omega \nabla u_W \cdot \nabla v = f \, v \diff x.
\end{equation}
We have now eased the continuity requirement on $u_W$.

We can also derive the weak solution from the energy minimization problem
\begin{equation}
    u_W = \argmin_{u \in H^1_g(\Omega)} \int_\Omega \frac{1}{2}(\nabla u)^2 - f \, u \diff x = \argmin_{u \in H^1_g(\Omega)} E(u, f).
\end{equation}
In order to see this, note that $H^1_g(\Omega)$ is known to be separable, which means that it is spanned by a countable set of basis functions $\{ \psi_i \}_{i = 1}^\infty$.
This means that we can write any $v \in H^1_g(\Omega)$ as
\begin{equation}
    v = \sum_i v_i \psi_i,
\end{equation}
and we then have
\begin{equation}
    \frac{\partial v}{\partial v_i} = \psi_i,
\end{equation}
or specifically in our case \( u_W = \sum_{i} u^W_i \psi_i \).
We then seek to find $u_W$ such that
\begin{equation}
    \frac{\partial E}{\partial u_W} = 0.
\end{equation}
This gives us
\begin{equation}
    \frac{\partial E}{\partial u_i^W} = \int_\Omega \nabla u_W \cdot \nabla \psi_i - f \, \psi_i \diff x = 0,
\end{equation}
which is exactly the weak formulation we had before.

As is a general strategy when approximating a function, we can also formulate a least squares problem based on the PDE.
Here we seek to minimize the functional
\begin{equation}
    E_{LS}(v, f) = \int_\Omega (-\Delta v - f)^2 \diff x.
\end{equation}
If $u_S$ is the strong solution, then we clearly have that $E_{LS}(u_S, f) = 0$.
If however we have a weak solution $u_W$, then we may have that $E_{LS}(u_W, f) = \infty$,
as $u_W \in H^1(\Omega)$, meaning that we only have control over the first derivative.
We must instead have that $u_{LS} \in H^2(\Omega)$, in order the get the proper regularity.
This means that we can write the least squares problem as
\begin{equation}
    u_{LS} = \argmin_{v \in H^2(\Omega)} E_{LS}(v, f).
\end{equation}

Solving this in the same manner as before, we find
\begin{align*}
    \frac{\partial E_{LS}}{\partial u_i^{LS}} &= \frac{\partial}{\partial u_i^{LS}} \int_\Omega (-\Delta u_{LS} - f)^2 \diff x \\
    &= \int_{\Omega} \frac{\partial }{\partial u_i^{LS}} (-\Delta u_{LS} - f)^2 \diff x \\
    &= 2 \int_{\Omega} (-\Delta u_{LS} - f) (-\Delta \psi_i) \diff x
\end{align*}