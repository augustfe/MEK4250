\subsubsection{Ciarlet's definition}
Ciarlet defined a finite element as a triplet $(T, V, D)$.
Here, $T$ is a bounded domain in $\mathbb{R}^d$, which is typically a polyhedron.
This corresponds to a section of the triangulation of the domain.
Next, $V = \{\psi_i\}_{i = 1}^n$ is a set of linearly independent basis functions defined on $T$.
Finally, $D = \{d_i\}_{i = 1}^n$ is a set of degrees of freedom, which are linear functionals defined on $V$.

We typically want each domain $T$ to be shape regular, i.e.\ such that they are all triangles or all quadrilaterals.
This allows us to perform the computations in a more efficient manner, where we are able to reuse a lot of the computations for each element.
Note that the basis functions $\psi_i$ are only defined locally on the element $T$, and thus they are not defined globally.

The degrees of freedom $d_i$ are what tie the local basis functions together, ensuring properties such as continuity of a certain order across the boundaries of the elements.
With the degrees of freedom, we have the associated \textit{nodal basis} $\{\phi_{i}\}_{i = 1}^n$, which are defined such that
\begin{equation}
    d_j(\phi_i) = \delta_{ij}.
\end{equation}
Finite elements are typically implemented directly through this nodal basis, where the coefficients of the basis functions are defined by the degrees of freedom.

\subsubsection{Functionals and function spaces}
Function spaces are vector spaces, where each element is a function.
Typically, we are looking at function spaces defined through various properties, such as the space of all continuous functions, or the space of all functions with a finite integral.
A functional on a vector space $V$ is then an object $L(\cdot)$, which takes in a vector $v \in V$ and returns a number.
For instance, from a set of coordinates $x_1, x_2, \ldots, x_n$, we can define a functional $L$ on a function space $V = \Span\{\phi_i\}_{i=1}^n$ such that
\begin{equation}
    L(v) = v(x_1) + v(x_2) + \cdots + v(x_n),
    \qquad
    v \in V
\end{equation}
or a set of functionals $L_j$ such that
\begin{equation}
    L_j(\phi_i) = \phi_i(x_j) = \delta_{ij},
    \qquad
    i = 1, 2, \ldots, n.
\end{equation}

Typical function spaces we are interested in the finite element method are Sobolev spaces $W^{k,p}(\Omega)$, which are defined as
\begin{equation}
    W^{k,p}(\Omega) =
    \left\{
        u :
        \left(
            \int_{\Omega}
            \sum_{i \leq k} \abs*{\frac{\partial^i u}{\partial x^i}}^p
            \diff x
        \right)^{1/p} < \infty
    \right\},
\end{equation}
essentially spaces where the function and its derivatives up to order $k$ are in $L^p(\Omega)$.

As far as I'm aware, we are really only interested in the spaces where $p = 2$, which we denote as $H^k(\Omega)$.
The $H$ is chosen after Hilbert, as these are in fact Hilbert spaces.
The inner product for $H^k(\Omega)$ is defined as
\begin{equation}
    (u, v)_{H^k(\Omega)}
    = (u, v)_k
    = \sum_{i \leq k} \int_{\Omega} \frac{\partial^i u}{\partial x^i} \frac{\partial^i v}{\partial x^i} \diff x.
\end{equation}
With the functionals we defined previously, if we choose the points $x_i$ to be along the edges of the elements, we can ensure that they are continuous across elements, which then gives us that the finite element space is a subspace of $H^1(\Omega)$.
Similarly, we can choose a different set of functionals such that we ensure higher order continuity, such as $H^2(\Omega)$.

\subsubsection{Reference element}
In order to illustrate the benefit of a reference element, we consider simply a number of segments of the real number line, with points $x_1 < x_2 < \cdots < x_n$.
Thus, each $T_i$ is simply the segment $[x_i, x_{i+1}]$ for $i = 1, 2, \ldots, n-1$.
On each segment, we have the linear Lagrange basis functions $\phi_i$, which are defined such that $\phi_i(x_j) = \delta_{ij}$.
This is illustrated with $n = 5$ points in \cref{fig:segments}.

\begin{figure}[htbp]
    \centering
    \begin{tikzpicture}
        \def\N{4} % Number of segments
        \foreach \x/\i in {0/1, 1.5/2, 2.5/3, 4/4, 5/5} {
            % Draw a filled circle of radius 2pt at (\x,0) and name that coordinate (x\i)
            \draw[fill=black] (\x,0) circle (2pt) coordinate (x\i);

            % Place the label "x_i" just below the same point (x\i)
            \node[below] at (x\i) {$x_{\i}$};
        }
        \draw[thick] (x1) -- (x\the\numexpr\N + 1\relax);

        \foreach \i in {1,...,\the\numexpr\N\relax} {
            \draw ($(x\i)+(0,1)$) -- (x\the\numexpr\i+1\relax);
            \draw (x\i) -- ($(x\the\numexpr\i+1\relax)+(0,1)$);
        }
    \end{tikzpicture}
    \caption{A number of segments of the real number line, with points $x_1 < x_2 < \cdots < x_n$, defining basis functions $\phi_i$ on each segment.\label{fig:segments}}
\end{figure}

When solving the finite element problem, we typically need to compute matrices $A$ and $M$ defined by
\begin{equation}
    A_{ij} = \int_{\Omega} \nabla \phi_i \cdot \nabla \phi_j \diff x,
    \qquad
    M_{ij} = \int_{\Omega} \phi_i \phi_j \diff x,
\end{equation}
where $A$ is the stiffness matrix and $M$ is the mass matrix.
With the basis functions we have here, these matrices will be incredibly sparse, as for instance $\phi_i$ is only non-zero on the segment $[x_{i-1}, x_{i+1}]$.
As such, we can beforehand say that for instance $A_{1,5}$ will be zero.

We then only need to compute the integrals for the non-zero entries, and we can additionally do this just on each segment, rather than the entire domain.
We are then left with the integrals of the form
\begin{equation}
    A_{ij}
    = \int_\Omega \phi'_i(x) \phi'_j(x) \diff x
    = \int_{x_i}^{x_{i+1}} \phi'_i(x) \phi'_j(x) \diff x
\end{equation}
for $j = i, i+1$.
As of now, we need to compute this for each $i$, however with a change of basis we can instead compute it just once for each segment, getting
\begin{equation}
    A_{j, k}^{(i)} = \int_{0}^{1} \ell'_j(X) \frac{\diff X}{\diff x} \ell'_k(X) \frac{\diff X}{\diff x} \frac{\diff x}{\diff X} \diff X,
\end{equation}
where $X$ is the reference element, and $dX/dx$ is the Jacobian of the transformation from the reference element to the segment.
In the linear case, $\frac{dX}{dx}$ is simply the length of the segment, and we can pull it out of the integral.
Constructing the total stiffness matrix then becomes a matter of computing the integral once for the reference element, and then adding the scaled contributions for each segment.
A similar argument holds for higher order spaces, mapping the physical element to the reference element, however the Jacobian will be more complicated, especially in the case of curved elements.

\subsubsection{Common elements in common spaces}
As mentioned previously, we are mostly interested in the spaces $L^2$, $H^1$ and $H^2$, as well as the spaces $H(\mathrm{div})$ and $H(\mathrm{curl})$.
Let's firstly describe the new spaces $H(\mathrm{div})$ and $H(\mathrm{curl})$.
They are given by
\begin{equation}
    \begin{split}
        H(\mathrm{div}) &= \left\{ u \in L^2(\Omega)^d : \nabla \cdot u \in L^2(\Omega) \right\},\\
        H(\mathrm{curl}) &= \left\{ u \in L^2(\Omega)^d : \nabla \times u \in L^2(\Omega) \right\}.
    \end{split}
\end{equation}

We are typically working with polynomial spaces, which are $C^\infty$ within their domain.
If we allow for discontinuities across the boundaries, we are in $L^2$.
If we enforce continuity across the boundaries, we are in $H^1$, and are typically using Lagrange elements.
If we enforce continuity of the first derivatives as well, we are in $H^2$, and are typically using Hermite elements.
For the spaces $H(\mathrm{div})$ and $H(\mathrm{curl})$, we typically use Raviart-Thomas elements and Nedelec elements respectively.

Raviart-Thomas elements ensure continuity of the normal component across boundaries, while jumps in the tangential component are allowed.
Similarly, as Stokes' Theorem relates curl to the tangential component, Nedelec elements ensure continuity of the tangential component across boundaries, while jumps in the normal component are allowed.