We consider the form of the convection-diffusion equation given by
\begin{equation}
    \begin{split}
        -\mu \Delta u + w \cdot \nabla u &= f \qquad \text{in } \Omega,\\
        u &= g \qquad \text{on } \partial\Omega,
    \end{split}
\end{equation}
where $\mu$ is the diffusion coefficient, $w$ is the velocity, and $f$ is the source term.
Here, I'm going to be working with $g = 0$, in order to simplify the computations.
This is a valid assumption, as we can apply lifting.

Multiplying by a test function $v$ and integrating by parts yields on the lhs
\begin{align*}
    \int_{\Omega} -\mu \Delta u\, v + w \cdot \nabla u \, v \diff x
    &= \int_\Omega \mu \nabla u \cdot \nabla v + w \cdot \nabla u\, v \diff x - \int_{\partial\Omega} \mu \frac{\partial u}{\partial n} v \diff s,
\end{align*}
which yields the weak form
\begin{equation}
    \int_{\Omega} \mu \nabla u \cdot \nabla v + \int_\Omega w \cdot \nabla u\, v \diff x = \int_{\Omega} f \, v \diff x + \int_{\partial\Omega} \mu \frac{\partial u}{\partial n} v \diff s.
\end{equation}
As $u$ is known along the boundary, we don't have to solve for $u$ here, and we can choose $v \in H^1_0(\Omega)$, which makes the boundary integral vanish.
We are then left with the problem of finding $u \in H^1_0(\Omega)$ such that
\begin{equation}
    a(u, v) = L(v)\qquad \forall v \in H^1_0(\Omega),
\end{equation}
where
\begin{equation}
    a(u, v) = \int_{\Omega} \mu \nabla u \cdot \nabla v + \int_\Omega (w \cdot \nabla u) \, v \diff x
    \quad\text{and}\quad
    L(v) = \int_{\Omega} f \, v \diff x.
\end{equation}

In order to look closer at when Lax--Milgram holds, we split the bilinear form into two parts $b$ and $c_w$, by % chktex 8
\begin{equation}
    a(u, v) = b(u, v) + c_w(u, v),
\end{equation}
where
\begin{equation}
    b(u, v) = \int_{\Omega} \mu \nabla u \cdot \nabla v
    \quad\text{and}\quad
    c_w(u, v) = \int_\Omega (w \cdot \nabla u) \, v.
\end{equation}
We then have that
\begin{align*}
    c_w(u, v) &= \int_\Omega (w \cdot \nabla u) \, v \diff x \\
    &= - \int_\Omega (w \cdot \nabla v)\, u \diff x - \int_{\Omega} \nabla \cdot w\, u\, v \diff x + \int_\Omega u\, v\, w \cdot n \diff s,
\end{align*}
where with Dirichlet conditions we can discard the last term.
Then, when $\nabla \cdot w = 0$, we have that
\begin{equation}
    c_w(u, v) = - \int_\Omega (w \cdot \nabla v)\, u \diff x = - c_w(v, u).
\end{equation}
This means that $c_w$ is skew-symmetric, which results in $c_w(u, u) = 0$, which means that we have
\begin{equation}
    a(u, u) = b(u, u) \geq \mu \abs{u}_1^2,
\end{equation}
which means that we have coercivity, as $\abs{\cdot}_1$ is equivalent to $\norm{\cdot}_1$ on $H^1_0(\Omega)$.

Next, we have
\begin{align*}
    a(u, v) &= \int_\Omega \mu \nabla u \cdot \nabla v + \int_\Omega (w \cdot \nabla u) \, v \diff x \\
    &\leq \mu \abs{u}_1 \abs{v}_1 + \abs{w}_\infty \abs{u}_1 \norm{v}_0 \\
    &\leq (\mu + \abs{w}_\infty C_\Omega) \abs{u}_1 \abs{v}_1.
\end{align*}
We therefore need $f \in H^{-1}_0$, $\nabla \cdot w = 0$, and $\abs{w}_\infty < \infty$ for Lax--Milgram to hold. % chktex 8
This then gives us the stability estimate
\begin{equation}
    \abs{u}_1 \leq \frac{\mu + C_\Omega \norm{w}_\infty}{\mu} \norm{f}_{-1}.
\end{equation}
If then $C_\Omega \norm{w}_\infty \gg \mu$, the stability constant will be large, and the solution will be unstable.

\subsubsection{Oscillations}
In order to explain the oscillations, consider the 1D case, with $w = 1$.
We then have the problem
\begin{equation}
    \begin{split}
        - u_x - \mu u_{xx} &= 0 \\
        u(0) = 0, \quad u(1) &= 1.
    \end{split}
\end{equation}
Using linear first order Lagrange elements, we seek $u \in H^1_{(0,1)}$ such that
\begin{equation}
    \int_{0}^{1} - u_x v + \mu u_x v_x \diff x = 0
\end{equation}
for all $v \in H^1_{(0,0)}$.

This discretization in 1D is equivalent with the finite difference scheme
\begin{equation}
    -\frac{\mu}{h^2}\left[
        u_{i+1} - 2u_i + u_{i-1}
    \right] - \frac{w}{2h} \left[
        u_{i+1} - u_{i-1}
    \right] = 0,
\end{equation}
with $u_0 = 0$ and $u_N = 1$.
In the extreme case where $\mu = 0$, the oscillations appear as each $u_{i + 1}$ is coupled with $u_{i - 1}$, but not with $u_i$.
Thus, for odd $N$, we will have that all even terms $u_{2i}$ will be equal to $u_0 = 0$, and all odd terms $u_{2i + 1}$ will be equal to $u_N = 1$.
This is the oscillations we observe.

If we however replace the central scheme for the derivative with a forward of backward scheme, we get
\begin{equation}
    \frac{w}{h} \left[
        u_{i+1} - u_i
    \right] = 0
    \quad\text{or}\quad
    \frac{w}{h} \left[
        u_i - u_{i-1}
    \right] = 0,
\end{equation}
which causes the oscillations to disappear, however the approximation is now only first order accurate.
We call this the upwind scheme, as it is biased in the direction of the flow.

If we discretize $u_x$ with a central scheme, and then add diffusion with a constant $\epsilon=h/2$, we get
\begin{equation}
    \frac{u_{i + 1} - u_{i - 1}}{2h} + \frac{h}{2} \frac{u_{i + 1} - 2u_i + u_{i - 1}}{h^2} = \frac{u_i - u_{i - 1}}{h},
\end{equation}
exactly the upwinding scheme!
In these cases, we are then actually solving the problem
\begin{equation}
    -(\mu + \epsilon)u_{xx} + w u_x = f.
\end{equation}

The choice of $\epsilon$ was crucial in order to stabilize the scheme, however it was chosen rather arbitrarily.
In order to avoid this, we turn to the Petrov--Galerkin method, where we allow the test function space to be different from the trial function space. % chktex 8
We choose the trial function space to be as before, however in the test space we utilize the basis functions
\begin{equation}
    L_j = N_j + \beta h (w \cdot \nabla N_j).
\end{equation}
This leaves us with the matrix $A_{ij}$ defined by
\begin{align*}
    A_{ij} &= a(N_i, L_j) = a(N_i, N_j + \beta h (w \cdot \nabla N_j)) \\
    &= \underbrace{
        \mu \nabla N_i \cdot \nabla N_j + (w \cdot \nabla N_i) N_j \diff x
    }_{\text{standard Galerkin}} \\
    &\quad + \beta h \underbrace{
        \int_\Omega \mu \nabla N_i \cdot \nabla(w \cdot \nabla N_j) \diff x
    }_{ = 0 \text{ for linear elements}}
    + \beta h \underbrace{
        \int_{\Omega} (w \cdot \nabla N_i) (w \cdot \nabla N_j) \diff x
    }_{\text{Artificial diffusion in $w$-direction}}.
\end{align*}
The rhs also changes, as
\begin{equation}
    L(L_j) = \int_\Omega f\, L_j \diff x
    = \int_\Omega f( N_j + \beta h (w \cdot \nabla N_j) ) \diff x,
\end{equation}
which means that we are adding diffusion in a consistent way to the rhs as well.