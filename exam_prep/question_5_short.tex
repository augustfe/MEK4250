The incompressible Navier--Stokes equation is given by % chktex 8
\begin{equation}
    \begin{split}
        \frac{\partial u}{\partial t} + u \cdot \nabla u = \frac{\nabla p}{\rho} + \nu \nabla^2u + f, \\
        \nabla \cdot u = 0.
    \end{split}
\end{equation}
Operator splitting refers to discretizing the equation in time before solving the equation in space.
Algebraic splitting on the other hand refers to splitting the equation in space first.

When operator splitting, we typically firstly discretize with a forward Euler scheme, or similar, which gives us an equation of the form
\begin{equation}
    \frac{u^{n+1} - u^n}{\Delta t} + u^n \cdot \nabla u^n = \frac{\nabla p^n}{\rho} + \nu \nabla^2 u^n + f^n.
\end{equation}
This has some issues, primarily the fact that we have no way of updating the pressure, as we have no expression for $p^{n+1}$.
In addition, we have no reason to assume that $\nabla \cdot u^{n+1} = 0$.

When algebraically splitting, we first find the weak form of the momentum equation, which is given by, with $\inp{\cdot}{\cdot}$ denoting the inner product in $L^2(\Omega)$,
\begin{equation}
    \inp*{\rho \frac{\partial u}{\partial t}}{v}
    + \inp{\rho u \cdot \nabla u}{v}
    - \inp{p}{\nabla \cdot v}
    + \inp{\nabla u}{\nabla v}
    = \inp{f}{v}
    + \inp{t_N}{v}_{\partial\Omega_N},
\end{equation}
while the weak form of the continuity equation is given by
\begin{equation}
    \inp{\nabla \cdot u}{q} = 0.
\end{equation}

We can then rewrite this in the discretized matrix form as
\begin{equation}
    \begin{split}
        M \frac{\partial u}{\partial t} + N(u)u + A u + B p &= f \\
        B^T u &= 0,
    \end{split}
\end{equation}
with the corresponding matrices as in the weak form.
This system however carries some issues, as it is non-linear, non-symmetric and indefinite, and there is in general no good method for solving it.
