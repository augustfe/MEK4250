\section{Discretization of Navier--Stokes} % chktex 8
Explain the difference between operator splitting and algebraic splitting in the context of the incompressible Navier--Stokes equations. % chktex 8
We remark that algebraic splitting is a term usually used for discretizations where the PDEs are discretized in space prior to time.
Show disadvantages for operator splitting schemes associated with boundary conditions.
Explain the advantage with operator splitting schemes.

\subsection{Operator splitting} % chktex 8
The incompressible Navier--Stokes equation is given by % chktex 8
\begin{equation}
    \begin{split}
        \frac{\partial u}{\partial t} + u \cdot \nabla u
        &= -\frac{\nabla p}{\rho} + \nu \nabla^2 u + f, \\
        \nabla \cdot u &= 0.
    \end{split}
\end{equation}

Operator splitting refers in this case to discretizing in time prior to space.
If we discretize simply with a forward Euler scheme, we end up with
\begin{equation}
    \frac{u^{n+1} - u^n}{\Delta t} + u^n \cdot \nabla u^n
    = -\frac{\nabla p^n}{\rho} + \nu \nabla^2 u^n + f^n,
\end{equation}
which rearranged gives us
\begin{equation}
    u^{n+1} = u^n + \Delta t \left(
        - u^n \cdot \nabla u^n
        -\frac{\nabla p^n}{\rho}
        + \nu \nabla^2 u^n
        + f^n
    \right).
\end{equation}
This is relatively simple, however it immediately raises a few issues.
For one, we get no expression for how we should update the pressure, i.e.\ what should $p^{n+1}$ be?
Secondly, we have no reason to assume that $\nabla \cdot u^{n+1} = 0$.

One way to overcome this is to introduce a tentative solution
\begin{equation}\label{eq:tent}
    u^* = u^n + \Delta t \left(
        - u^n \cdot \nabla u^n
        -\frac{\nabla p^n}{\rho}
        + \nu \nabla^2 u^n
        + f^n
    \right),
\end{equation}
then saying that the real solution $u^{n+1}$ should satisfy
\begin{equation}\label{eq:real}
    u^{n + 1} = u^n + \Delta t \left(
        - u^n \cdot \nabla u^n
        -\frac{\nabla p^{n+1}}{\rho}
        + \nu \nabla^2 u^n
        + f^n
    \right).
\end{equation}
Subtracting \cref{eq:tent} from \cref{eq:real}, we get
\begin{equation}\label{eq:correction}
    u^{n + 1} - u^* = -\frac{\Delta t}{\rho} \nabla \left( p^{n+1} - p^n \right).
\end{equation}
As we should have $\nabla \cdot u^{n+1} = 0$, taking the divergence of \cref{eq:correction} gives us
\begin{equation}
    \nabla \cdot u^* = \frac{\Delta t}{\rho} \nabla^2 \left( p^{n + 1} - p^{n} \right).
\end{equation}
Letting now $\phi = p^{n + 1} - p^{n}$, we can write this as
\begin{equation}
    \nabla^2 \phi = \frac{\rho}{\Delta t} \nabla \cdot u^*,
\end{equation}
which we recognize as a Poisson problem, as $u^*$ is known.

This basically gives us all the ingredients we need for a scheme, which we summarize as
\begin{enumerate}
    \item Compute the tentative velocity
        \begin{equation}
            u^*  = u^n + \Delta t \left(
                - u^n \cdot \nabla u^n
                -\frac{\nabla p^n}{\rho}
                + \nu \nabla^2 u^n
                + f^n
            \right).
        \end{equation}

    \item Solve the Poisson problem
        \begin{equation}\label{eq:possion}
            -\nabla^2 \phi = -\frac{\rho}{\Delta t} \nabla \cdot u^*
        \end{equation}
        in order to find the update for the pressure.

    \item Update the velocity by
        \begin{equation}
            u^{n + 1} = u^* - \frac{\Delta t}{\rho} \nabla \phi.
        \end{equation}

    \item Update the pressure by
        \begin{equation}
            p^{n + 1} = p^n + \phi.
        \end{equation}
\end{enumerate}
This scheme is rather simple when stated like this, however there are some unadressed issues.
Firstly is the issue of boundary trouble.
As we need to solve a Poisson problem in the pressure update, we require boundary conditions along the entire boundary, even though this isn't strictly necessary in the original problem.

Another approach is to use an implicit scheme, given where we get the tentative velocity by
\begin{equation}
    \frac{u^* - u^n}{\Delta t} + u^* \cdot \nabla u^* = -\frac{\nabla p^n}{\rho} + \nu \nabla^2 u^* + f^{n + 1}.
\end{equation}
Rearranging this again leads to
\begin{equation}
    u^* - \Delta t \left(
        (-u^* \cdot \nabla u^*)
        - \frac{1}{\rho} \nabla p^n
        + \nu \nabla^2 u^*
    \right)
    = u^n + \Delta t f^{n + 1}.
\end{equation}
The term $u^* \cdot \nabla u^*$ is problematic as it is non-linear, however we can apply a typical linearization technique by replacing it with $u^n \cdot \nabla u^*$ to get
\begin{equation}
    u^* - \Delta t \left(
        (-u^n \cdot \nabla u^*)
        - \frac{1}{\rho} \nabla p^n
        + \nu \nabla^2 u^*
    \right)
    = u^n + \Delta t f^{n + 1}.
\end{equation}
What we really want is however
\begin{equation}
    u^{n + 1} - \Delta t \left(
        (-u^{n} \cdot \nabla u^{n + 1})
        - \frac{1}{\rho} \nabla p^n
        + \nu \nabla^2 u^{n + 1}
    \right)
    = u^n + \Delta t f^{n + 1}.
\end{equation}
Subtracting the first equation from the second again now gives us a more complicated expression.

For simplicity, we introduce the convection-diffusion operator
\begin{equation}
    s(u^c) = \Delta t \left( -u^n \cdot \nabla u^c + \nu \nabla^2 u^c \right).
\end{equation}
With
\begin{equation}
    u^{n+1} = u^* + u^c,
\end{equation}
we can now write
\begin{equation}
    \begin{split}
        u^c - s(u^c) + \frac{\Delta t}{\rho}\nabla \phi &= 0 \\
        \nabla \cdot u^c &= - \nabla \cdot u^*.
    \end{split}
\end{equation}
So far, we haven't gotten any closer to our deisred goal, as it it just as hard as the original equations.
However, note that we are using a first order approximation to the time derivative, and as the leading term in $s(u^c)$ is first order, we can drop it while still having a first order approximation.
We then get
\begin{equation}
    \begin{split}
        u^c + \frac{\Delta t}{\rho} \nabla \phi &= 0 \\
        \nabla \cdot u^c &= -\nabla \cdot u^*,
    \end{split}
\end{equation}
where we can again rewrite the first equation as
\begin{equation}
    -\nabla^2 \phi = - \frac{\rho}{\Delta t} \nabla \cdot u^*.
\end{equation}

This second approach amounts to the Incremental Pressure Correction Scheme (IPCS), which consists of the four steps
\begin{enumerate}
    \item Compute the tentative velocity
        \begin{equation}
            u^* - s(u^*) + \frac{\Delta}{\rho}\nabla p^n = f^{n + 1}.
        \end{equation}

    \item Solve the Poisson problem for the pressure
        \begin{equation}
            -\nabla^2 \phi = - \frac{\rho}{\Delta t} \nabla \cdot u^*
        \end{equation}

    \item Update the velocity
        \begin{equation}
            u^{n + 1} = u^* - \frac{\Delta t}{\rho} \nabla \phi
        \end{equation}

    \item Update the pressure
        \begin{equation}
            p^{n + 1} = p^n + \phi
        \end{equation}
\end{enumerate}

\subsection{Issues with operator splitting}
Operator splitting introduces trouble along the boundary.
NS in itself, in 3D, requires 3 conditions in every point at the boundary.
IPCS however requries 4 conditions along the boundary, 3 for the tentative velocity, and 1 for the Poisson equation.

We can derive boundary conditions for $\phi$ in two ways:
\begin{enumerate}
    \item From the scheme
        \begin{enumerate}
            \item \[
                    u^{n + 1} = u^* - \frac{\Delta t}{\rho} \nabla \phi
                \]

            \item As $u^{n+1}$ and $u^*$ have the same BCs, we obtain homogenous Neumann conditions for $\phi$, as
                \begin{equation}
                    \nabla \phi \cdot n = \frac{\rho}{\Delta t} (u^{n + 1} - u^*) \cdot n = 0
                \end{equation}
        \end{enumerate}

    \item From the Navier--Stokes equations we have that % chktex 8
        \begin{enumerate}
            \item \[
                    \nabla p^n = - \rho\left(
                        \frac{\partial u^n}{\partial t} + u^n \cdot \nabla u^n
                    \right) + \mu\nabla^2 u^n + f
                \]

            \item As $\phi = p^{n+1} - p^n$ and \( p^{n + 1} \neq p^n \) we obtain a non-homogeneous condition.
        \end{enumerate}
\end{enumerate}
In conclusion, we arrive at two different conditions which both seem reasonable.
The difference between the two is first order.

\subsection{Algebraic splitting}
Here, we firstly seek the weak form of the NS equations.
We use \( \langle \cdot, \cdot \rangle \) to denote the \(L^2\)-inner product.
Multiplying the momemtum equation by a test and integrating by parts then yields
\begin{equation}
    \langle \rho \frac{\partial u}{\partial t} , v \rangle
    + \langle \rho u \cdot \nabla u, v \rangle
    - \langle p, \nabla \cdot v \rangle
    + \langle \nabla u, \nabla v \rangle
    = \langle f, v \rangle + \langle t_N, v \rangle_{\Gamma_N},
\end{equation}
doing the same with the continuity equation yields
\begin{equation}
    \langle \nabla \cdot u, q \rangle = 0.
\end{equation}
