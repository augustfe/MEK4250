The Poisson problem with a variable coefficient is given by
\begin{equation}
    \begin{split}
        -\nabla \cdot \left( \kappa \nabla u\right) &= f \qquad \text{in } \Omega,\\
        u &= g \qquad \text{on } \partial\Omega_D, \\
        \kappa \frac{\partial u}{\partial n} &= h \qquad \text{on } \partial\Omega_N,
    \end{split}
\end{equation}
where we assume that $\kappa$ is positive, symmetric and bounded.

In order to get the weak form of this problem, we multiply by a test function $v$ and integrate by parts, which yields
\begin{align*}
    \int_\Omega -\nabla \cdot \left( \kappa \nabla u \right) v \diff x
    &= \int_\Omega f \, v \diff x \\
    \int_\Omega \kappa \nabla u \cdot \nabla v \diff x &= \int_{\Omega} f \, v \diff x + \int_{\partial\Omega} \kappa \frac{\partial u}{\partial n} \, v \diff s.
\end{align*}
We can split the boundary integral into two parts, one for the Dirichlet boundary and one for the Neumann boundary:
\begin{align*}
    \int_{\partial\Omega} \kappa \frac{\partial u}{\partial n} \, v \diff s
    &= \int_{\partial\Omega_D}  \kappa \frac{\partial u}{\partial n} \, v \diff s + \int_{\partial\Omega_N} h \, v \diff s \\
    &= \int_{\partial\Omega_D} \kappa \frac{\partial u}{\partial n} \, 0 \diff s + \int_{\partial\Omega_N} h \, v \diff s.
\end{align*}
As the solution is known along the Dirichlet boundary, we can choose $v \in H^1_{0,D}(\Omega)$, which makes the boundary integral there vanish.
This leaves us with the weak form of the Poisson problem, which amounts to finding $u \in H_{g, D}^1(\Omega)$ such that
\begin{equation}
    \int_{\Omega} \kappa \nabla u \cdot \nabla v \diff x = \int_{\Omega} f \, v \diff x + \int_{\partial\Omega_N} h \, v \diff s
    \qquad \forall v \in H^1_{0,D}(\Omega).
\end{equation}
We then get the finite element formulation by discretizing the function space.

For Lax--Milgram to hold, we need to show that: % chktex 8
\begin{align}
    a(u, v) &\leq C_1 \norm{u}_V \norm{v}_V, \qquad \forall u, v \in V, \\
    a(u, u) &\geq C_2 \norm{u}_V^2, \qquad \forall u \in V, \\
    L(v) &\leq C_3 \norm{v}_V, \qquad \forall v \in V,
\end{align}
where $a(u, v)$ is the bilinear form in the weak form, and $L(v)$ is the rhs.

We firstly have
\begin{equation}
    a(u, v) = (\kappa \nabla u, \nabla v)_{L^2} \leq \kappa_{\max} \abs{u}_1 \abs{v}_1 \leq \kappa_{\max} \norm{u}_1 \norm{v}_1,
\end{equation}
where we've used the fact that $\kappa$ is bounded and the Cauchy--Schwarz inequality.

Next, we have
\begin{equation}
    a(u, u) = (\kappa \nabla u, \nabla u)_{L^2} \geq \kappa_{\min} \abs{u}_1^2 \geq C_2 \norm{u}_1^2,
\end{equation}
where we've used the fact that $\kappa$ is bounded below and positive.
We've also used the fact that the $H^1$-norm is equivalent to the $H^1$-semi-norm on $H^1_0$, applying lifting if necessary.

Finally, we have
\begin{align*}
    L(v) &= \int_{\Omega} f \, v \diff x + \int_{\partial\Omega_N} h \, v \diff s \\
    &\leq \norm{f}_{L^2(\Omega)} \norm{v}_{L^2(\Omega)} + \norm{h}_{L^2(\partial\Omega_N)} \norm{v}_{L^2(\partial\Omega_N)} \\
    &\leq C_3 \norm{v}_V,
\end{align*}
assuming we can bound $\norm{v}_{L^2(\partial\Omega_N)}$ by $\norm{v}_1$, and that $f$ and $h$ are bounded.

Extending the equation to the convection-diffusion equation, we simply add the convection term
\begin{equation}
    w \cdot \nabla u = 0
\end{equation}
to the strong form of the equation, which alters the weak form to
\begin{equation}
    a(u, v) = \int_{\Omega} \kappa \nabla u \cdot \nabla v \diff x + \int_{\Omega} w \cdot \nabla u\, v \diff x = \int_{\Omega} f \, v \diff x + \int_{\partial\Omega_N} h \, v \diff s,
\end{equation}
where we've made the assumption that $w$ is bounded.

With the energy norm defined as $\norm{u}_E^2 = a(u, u)$, we have
\begin{align*}
    \norm{u - u_h}_E^2 &= a(u - u_h, u - u_h) \\
    &= a(u - u_h, u - v + v - u_h) \\
    &= a(u - u_h, u - v) + a(u - u_h, v - u_h) \\
    &= a(u - u_h, u - v) + 0 \\
    &\leq \norm{u - u_h}_E \norm{u - v}_E
\end{align*}
which gives us that
\begin{equation}
    \norm{u - u_h}_E \leq \norm{u - v}_E,
\end{equation}
where the choice of $v \in V$ is arbitrary.
Here, we've used Galerkin orthogonality, which states that the error is orthogonal to the finite element space.
Next, choosing $v$ to be the polynomial interpolant of $u$ of degree $t$, we have
\begin{equation}
    \norm{u - I_h u}_E^2 = a(u - I_h u, u - I_h u) \leq \frac{k_1}{1 + C_P} \norm{u - I_h u}_1^2 \leq \frac{k_1}{1 + C_P} (B h^t)^2 \norm{u}_{t+1}^2,
\end{equation}
which yields the error estimate
\begin{equation}
    \norm{u - u_h}_E \leq C h^t \norm{u}_{t+1}.
\end{equation}

With this, we can estimate the convergence rate with elements of a given degree, simply by changing the mesh size $h$, and comparing the error repeatedly.
