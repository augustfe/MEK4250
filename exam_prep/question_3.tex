\section{Discretization of Stokes}
Derive a proper variational formulation of the Stokes problem.
Discuss the four Brezzi conditions that are needed for a well-posed continuous problem.
Explain why oscillations might appear in the pressure for some discretization techniques.
Present expected approximation properties for mixed elements that satisfy the inf-sup condition, and discuss a few examples like e.g.\ Taylor--Hood, Mini, and Crouzeix--Raviart. % chktex 8
Discuss also how one might cirumvent the inf-sup condition by stabilization.

\subsection{Weak form of Stokes problem}
The Stokes problem describes the flow of a slowly moving viscous incompressible Newtonian fluid.
Let $u : \Omega \to \mathbb{R}^n$ be the fluid field, and $p: \Omega \to \mathbb{R}$ be the fluid pressure.
Stokes problem can then be written as
\begin{equation}
    \begin{split}
        -\Delta u + \nabla p &= f \quad \text{in } \Omega, \\
        \nabla \cdot u &= 0 \quad \text{in } \Omega, \\
        u &= g \quad \text{on } \partial\Omega_D, \\
        \frac{\partial u}{\partial n} - p \, n &= h \quad \text{on } \partial\Omega_N.
    \end{split}
\end{equation}
Here, $f$ is the body forcce, $\partial\Omega_D$ and $\partial\Omega_N$ are the Dirichlet and Neumann boundaries, respectively.
Additionally, $g$ is the prescribed fluid velocity on the Dirichlet boundary, and $h$ is the surface force or stress on the Neumann boundary.

The presence of the Dirichlet and Neumann boundary conditions lead to a well-posed problem, so long as neither are empty.
If the Dirichlet condition if empty, the velocity is only determined up to a constant.
If the Neumann condition is empty, the pressure is only determined up to a constant.

We being by setting up the weak form of the Stokes problem.
Setting up the weak form amounts to the following steps:
\begin{enumerate}
    \item Multiply by test functions and integrate of the domain $\Omega$.

    \item Integration by parts, and apply Gauss--Green's lemma. % chktex 8

    \item Apply the boundary conditions.
\end{enumerate}
This leads us to initially have, multiplying the first equation by a test function $v$ and the second equation by a test function $q$,
\begin{equation}
    \begin{split}
        \int_\Omega (-\Delta u + \nabla p) \cdot v \, \diff x &= \int_\Omega f \cdot v \, \diff x \\
        \int_\Omega (\nabla \cdot u) q \, \diff x &= 0.
    \end{split}
\end{equation}
This is however not ideal, as we would currently be requiring that $u \in H_{g, D}^2$ and $p \in H^1$.
We therefore apply integration by parts, which yields
\begin{equation}
    \int_\Omega (-\Delta u + \nabla p) \cdot v \, \diff x
    = \int_\Omega \nabla u : \nabla v + p (\nabla \cdot v) \, \diff x
    + \int_{\partial\Omega} \left(\frac{\partial u}{\partial n} - p n\right) \cdot v \, \diff s.
\end{equation}
We then consider the boundary conditions, which yields
\begin{align*}
    \int_{\partial\Omega} \left(\frac{\partial u}{\partial n} - p n\right) \cdot v \, \diff s
    &= {
        \int_{\partial\Omega_D} \left(\frac{\partial u}{\partial n} - p n\right) \cdot v \, \diff s
        + \int_{\partial\Omega_N} \left(\frac{\partial u}{\partial n} - p n\right) \cdot v \, \diff s
    } \\
    &= {
        \underbrace{
            \int_{\partial\Omega_D} \left(\frac{\partial u}{\partial n} - p n\right) \cdot 0 \, \diff s
        }_{\text{As we choose } v \in H^1_{0, D}}
        + \int_{\partial\Omega_N} h \cdot v \, \diff s
    } \\
    &= \int_{\partial\Omega_N} h \cdot v \, \diff s.
\end{align*}
Defining
\begin{equation}
    \begin{split}
        a(u, v) &= \int_{\Omega} \nabla u : \nabla v \, \diff x \\
        b(v, p) &= \int_{\Omega} p (\nabla \cdot v) \, \diff x \\
        L(v) &= \int_{\Omega} f \cdot v \, \diff x + \int_{\partial\Omega_N} h \cdot v \, \diff x
    \end{split}
\end{equation}
we can rewrite the weak form of Stokes problem succinctly as:
Find $(u, p) \in V \times Q$ such that for all $(v, q) \in \hat{V} \times \hat{Q}$
\begin{equation}
    \begin{split}
        a(u, v) - b(v, p) &= L(v), \\
        b(u, q) &= 0.
    \end{split}
\end{equation}

The finite element formulation follows directly from this:
Find $u_h \in V_{g, h}$ and $p_h \in Q_h$ such that
\begin{equation}\label{eq:stokes-fem}
    \begin{split}
        a(u_h, v_h) + b(p_h, v_h) &= L(v_h) \quad \forall v_h \in V_{0, h}, \\
        b(q_h, u_h) &= 0 \quad\qquad \forall q_h \in Q_h.
    \end{split}
\end{equation}

\subsection{Brezzi conditions}
For a saddle point problem of the form \cref{eq:stokes-fem} to be well-posed, we require that four conditions are satisfied.
\begin{enumerate}
    \item Boundedness of $a$:
        \begin{equation}
            a(u_h, v_h) \leq C_1 \norm{u_h}_{V_h} \norm{v_{h}}_{V_h}
            \quad
            \forall u_h, v_h \in V_h.
        \end{equation}

    \item Boundedness of $b$:
        \begin{equation}
            b(u_h, q_h) \leq C_2 \norm{u_h}_{V_h} \norm{q_{h}}_{Q_h}
            \quad
            \forall {
                u_h \in V_h,
                q_h \in Q_h.
            }
        \end{equation}

    \item Coersivity of $a$:
        \begin{equation}
            a(u_h, u_h) \geq C_3 \norm{u_h}_{V_h}^2
            \quad
            \forall u_h \in Z_h,
        \end{equation}
        where $Z_h = \{ u_h \in V_h | b(u_h, q_h) = 0 \forall q_h \in Q_h \}$.

    \item ``Coercivity'' of $b$:
        \begin{equation}
            \sup_{u_h \in V_h} \frac{
                b(u_h, q_h)
            }{
                \norm{u_h}_{V_h}
            } \geq C_4 \norm{q_h}_{Q_h}
            \quad
            \forall q_h \in Q_h.
        \end{equation}
\end{enumerate}
The first three conditions are easily verified for Stokes problem, while the last one is difficult unless the elements are designed specifically to meet this condition.