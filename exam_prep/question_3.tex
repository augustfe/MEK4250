\section{Discretization of Convection-Diffusion}
Derive a proper variational formulation of the convection-diffusion problem.
Derive sufficient conditions that make the problem well-posed.
Discuss why oscillations appear for standard Galerkin methods and show how SUPG methods resolve these problems.
Discuss also approximation properties in light of Cea's lemma.

\subsection{Weak form of the Convection-Diffusion equation}
Here, we consider the convection-diffusion equation given by
\begin{equation}
    \begin{split}
        -\mu \Delta u + w \cdot \nabla u &= f, \quad \text{in } \Omega,\\
        u &= g, \quad \text{on } \partial\Omega,
    \end{split}
\end{equation}
assuming Dirichlet conditions on the whole boundary.

In order to derive the weak form, we follow the steps:
\begin{enumerate}
    \item Multiply the equation with a test function $v$ and integrate.

    \item Integrate by parts, and apply Gauss--Green's lemma. % chktex 8

    \item Apply the boundary conditions.
\end{enumerate}
The first step gives us
\begin{equation}
    \int_\Omega -\mu \Delta u \, v + w \cdot \nabla u \, v \, \diff x = \int_\Omega f \, v \, \diff x.
\end{equation}
Then, we use integration by parts on the first term in order to ease the requirement of $u \in H^2(\Omega)$ to $u \in H^1(\Omega)$, while strengthening the requirement on $v$ from $v \in L_2(\Omega)$ to $v \in H^1(\Omega)$.
This gives us
\begin{equation}
    \int_{\Omega} \mu \nabla u \cdot \nabla v  + w \cdot \nabla u \, v \, \diff x = \int_{\Omega} f \, v \, \diff x + \int_{\partial\Omega} \mu \frac{\partial u}{\partial n} v \, \diff s.
\end{equation}
Next, we consider the boundary term.
As the solution is known on the boundary, we need to solve for $u$ on the boundary, and can therefore choose $v \in H_0^1(\Omega)$, such that
\begin{equation}
    \int_{\partial\Omega} \mu \frac{\partial u}{\partial n} v \, \diff s
    = \int_{\partial\Omega} \mu \frac{\partial u}{\partial n} 0 \, \diff s = 0,
\end{equation}
effectively removing the boundary term from our formulation.

We can now write the weak form of the convection-diffusion equation as
\begin{equation}
    \int_{\Omega} \mu \nabla u \cdot \nabla v + w \cdot \nabla u \, v \, \diff x = \int_{\Omega} f \, v \, \diff x.
\end{equation}
The bilinear form is then given by
\begin{equation}
    a(u,v) = \int_{\Omega} \mu \nabla u \cdot \nabla v + w \cdot \nabla u \, v \, \diff x,
\end{equation}
and the linear form is given by
\begin{equation}
    L(v) = \int_{\Omega} f \, v \, \diff x.
\end{equation}
The weak form of the problem is then, find $u \in V$ such that
\begin{equation}
    a(u,v) = L(v), \quad \forall v \in V.
\end{equation}
% Here, we assume that $\kappa$ is symmetric, positive and bounded, and that $w$ is bounded.

\subsection{Well-posedness}
For well-posedness, we rely on the Lax--Milgram theorem, meaning we have to find sufficient conditions such that: % chktex 8
\begin{enumerate}
    \item $a(u, u) \geq \alpha \norm{u}^2_V$ for some $\alpha > 0$ and all $u \in V$.
    \item $a(u, v) \leq \beta \norm{u}_V \norm{v}_V$ for some $\beta > 0$ and all $u, v \in V$.
    \item $L(v) \leq D \norm{v}_V$ for some $D > 0$ and all $v \in V$.
\end{enumerate}

Here, we'll work through the conditions in reverse order, adding conditions as we go.
For the third condition, we simply have by Cauchy--Schwarz % chktex 8
\begin{equation}
    L(v) = \int_\Omega f \, v \, \diff x \leq \norm{f}_{L^2} \norm{v}_{L^2} \leq \norm{f}_{L^2} \norm{v}_1,
\end{equation}
showing that we require that $f \in L^2(\Omega)$ in order to satisfy the third condition.
For the second condition, we apply lifting to $u$, such that we can use Poincaré's inequality.
We then have
\begin{align*}
    a(u, v)
    &= \int_{\Omega} \mu \nabla u \cdot \nabla v + w \cdot \nabla u \, v \, \diff x \\
    &\leq \mu \abs{u}_1 \abs{v}_1 + \norm{w}_{L^\infty} \abs{u}_1 \norm{v}_{L^2} \\
    &\leq \mu \abs{u}_1 \abs{v}_1 + C \norm{w}_{L^\infty} \abs{u}_1 \abs{v}_1 \\
    &\leq \left( \mu + C \norm{w}_{L^\infty} \right) \abs{u}_1 \abs{v}_1.
\end{align*}
As we've applied lifting to $u$, we can assume that $u \in H^1_0(\Omega)$, such that $\abs{u}_1$ is an equivalent norm to $\norm{u}_1$.

Finally, for the first condition, we write
\begin{equation}
    a(u, v) = b(u, v) + c_w(u, v),
\end{equation}
where $b(u, v) = \int_{\Omega} \mu \nabla u \cdot \nabla v$ and $c_w(u, v) = \int_{\Omega} w \cdot \nabla u \, v$.
For $b$, we already have
\begin{equation}
    b(u, u) = \int_{\Omega} \mu (\nabla u)^2 \, \diff x = \mu \abs{u}_1^2 \geq \frac{\mu}{1 + C} \norm{u}^2_2,
\end{equation}
as
\begin{equation}
    \norm{u}^2_2 = \abs{u}_1^2 + \norm{u}_{L^2}^2 \leq (1 + C) \abs{u}_1^2.
\end{equation}
$c_w$ is a bit more involved, however we start with integration by parts in order to get
\begin{align*}
    c_w(u, v)
    &= \int_{\Omega} w \cdot \nabla u \, v \, \diff x \\
    &= -\int_{\Omega} w \cdot \nabla v \, u \diff x - \int_\Omega \nabla \cdot w \, u \, v \, \diff x + \int_{\partial\Omega} w \cdot n \, u \, v \, \diff s.
\end{align*}
The boundary term vanishes as we've applied lifting, and if we assume that $\nabla \cdot w = 0$ such that we have incompressibility, we are left with
\begin{equation}
    c_w(u, v) = \int_{\Omega} w \cdot \nabla u \, v \, \diff x = -\int_{\Omega} w \cdot \nabla v \, u \diff x = -c_w(v, u).
\end{equation}
$c_w$ is then skew-symmetric, such that we have
\begin{equation}
    c_w(u, u) = -c_w(u, u) = 0.
\end{equation}
The convection-diffusion equation is then well-posed if we assume that $w$ is bounded and incompressible, such that $\nabla \cdot w = 0$.

\subsection{Oscillations in Galerkin methods}
In order to illustrate the oscillations, we consider a simplified scenario in one dimension, where we set $w = -1$.
We then have the equation
\begin{equation}
    \begin{split}
        -\mu u_{xx} - u_x &= 0, \\
        u(0) &= 0, \\
        u(1) &= 1.
    \end{split}
\end{equation}
The variational problem is then, find $u \in H^1_0(0, 1)$ such that
\begin{equation}
    \int_{0}^{1} \mu u_x v_x - u_x v \, \diff x = 0, \quad \forall v \in H^1_0(0, 1).
\end{equation}
Using first order Lagrange elements, we have that the discretization is equivalent to the central finite difference scheme
\begin{equation}
    -\frac{\mu}{h^2} \left[
        u_{i+1} - 2u_i + u_{i-1}
    \right]
    - \frac{w}{2h} \left[
        u_{i+1} - u_{i-1}
    \right] = 0, \quad i = 1, \ldots, N-1,
\end{equation}
which for $\mu = 0$ reduces to $u_{i + 1} = u_{i - 1} = \ldots$ and $u_{i + 2} = u_i = u_{i - 2} = \ldots$.
This is the cause of the oscillations, as if $N$ is odd, then we'll have that all terms of the form $u_{2i} = 0$, while $u_{2i + 1} = 1$, determined by the boundary conditions.

In order to get rid of these oscillations, we can apply upwinding, which amounts to using the approximations
\begin{equation}
    \begin{split}
        \frac{du}{dx}(x_i) &= \frac{1}{h} \left[
            u_{i+1} - u_{i}
        \right] \quad \text{if } w < 0, \\
        \frac{du}{dx}(x_i) &= \frac{1}{h} \left[
            u_{i} - u_{i-1}
        \right] \quad \text{if } w > 0.
    \end{split}
\end{equation}
The oscillations will then dissapear, however we are now using a first order scheme, rather than the second order scheme we had with the central finite difference scheme.
One way to look at this is by noting that
\begin{equation}
    \frac{u_i - u_{i-1}}{h} = \frac{u_{i + 1} - u_{i - 1}}{2h} + \frac{h}{2} \frac{-u_{i + 1} + 2u_i - u_{i - 1}}{h^2},
\end{equation}
as this shows that we are adding a diffusion term with coefficient $\varepsilon = \frac{h}{2}$ to the equation.

This shows that we are then actually solving the problem
\begin{equation}
    -(\mu + \varepsilon) u_{xx} - u_x = f,
\end{equation}
as opposed to the original problem.

\subsection{Streamline diffusion/Petrov--Galerkin} % chktex 8
Streamline diffusion/Petrov--Galerkin (SUPG) methods are a more general and ordered way of dealing with the oscillations we saw in the previous section. % chktex 8
We then add the diffusion in a consistent way, such that we aren't changing the solution as $h \to 0$.

The Petrov--Galerkin method is maybe unsurprisingly very similar to the standard Galerkin method, given by: % chktex 8
Find $u_h \in V_{h,g}$ such that
\begin{equation}
    a(u_h, v_h) = L(v_h), \quad \forall v_h \in W_{h, 0},
\end{equation}
where the difference is that the test space is now different from the trial space.

In matrix form, the Galerkin formulation yields
\begin{equation}
    A_{ij} = a(N_i, N_j) = \int_{\Omega} \mu \nabla N_i \cdot \nabla N_j + w \cdot \nabla N_i \, N_j \, \diff x,
\end{equation}
while the Petrov--Galerkin formulation yields % chktex 8
\begin{equation}
    A_{ij} = a(N_i, L_j) = \int_{\Omega} \mu \nabla N_i \cdot \nabla L_j + w \cdot \nabla N_i \, L_j \, \diff x,
\end{equation}
where $L_j$ is the test function.
Choosing the functions $L_j$ carefully is the key to adding diffusion in a consistent way.

We let $L_j = N_j + \beta h (w \cdot \nabla N_j)$.
This gives us
\begin{align*}
    A_{ij} &= a(N_i, N_j + \beta h (w \cdot \nabla N_j)) \\
    &= \int_{\Omega} \mu \nabla N_i \cdot (N_j + \beta h w \cdot \nabla N_j) \, \diff x + \int_{\Omega} w \cdot \nabla N_i (N_j + \beta h (w \cdot \nabla N_j)) \, \diff x \\
    &= \underbrace{
        \int_{\Omega} \mu \nabla N_i \cdot N_j \, \diff x
        + \int_{\Omega} w \cdot \nabla N_i \, N_j \, \diff x
    }_{\text{Standard Galerkin term}} \\
    &\quad + \underbrace{
        \beta h
        \int_{\Omega} \mu \nabla N_i \cdot \nabla (w \cdot \nabla N_j) \, \diff x
    }_{\text{Vanishes for linear elements}}
    + \underbrace{
        \beta h
        \int_{\Omega} (w \cdot \nabla N_i) (w \cdot \nabla N_j) \, \diff x
    }_{\text{Artificial diffusion in $w$ direction}}.
\end{align*}
The right hand side also changes, denoting the linear form now as $b(L_j)$, such that we have
\begin{equation}
    b(L_j)
    = \int_{\Omega} f L_j \, \diff x
    = \int_{\Omega} f (N_j + \beta h (w \cdot \nabla N_j)) \, \diff x.
\end{equation}
We are in other words changing both sides of the equation, such that the artifical diffusion is consistent.

\subsection{Cea's lemma}
Cea's lemma states that, given the conditions for Lax--Milgram are satisfied, we have % chktex 8
\begin{equation}
    \norm{u - u_h}_{V} \leq \frac{C B}{\alpha} h^t \norm{u}_{t + 1}.
\end{equation}
where $B$ comes the polynomial approximation properties, and $\alpha$ and $C$ are the constants from the Lax--Milgram theorem. % chktex 8
For convection-dominated problems, $\frac{C}{\alpha}$ is large, which causes poor approximation on coarse grids.

In order to get improved error estimates for the SUPG method, we utilize an alternative norm, given by
\begin{equation}
    \norm{u}_{sd} = (h \norm{w \cdot \nabla u}^2 + \mu \abs{\nabla u}^2)^{1/2}.
\end{equation}
Given that the conditions for Lax--Milgram still hold, solving the SUPG problem on a finite element space of order 1 gives us % chktex 8
\begin{equation}
    \norm{u - u_h}_{sd} \leq C h^{3/2} \norm{u}_2.
\end{equation}
The norm $\norm{u}_{sd}$ is called the \textit{streamline diffusion} norm.

The proof for this is very involved, and I'd rather not to too much into detail about it.
One thing to note however is that this error bound is independent of the convection velocity, which is a big improvement over the standard Galerkin method.